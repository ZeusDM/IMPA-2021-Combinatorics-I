\lecture{4}{January 07, 2021}{}
\begin{defn}
	Let $r(G, H)$ denote the minimum $n$ such that, for every coloration $c \colon E(K_n) \to \{R, B\}$, there must exist a red $G$ or a blue $H$.
\end{defn}

\begin{prop}
	\[
		\chi(G) \le \Delta(G) + 1.
	\]
\end{prop}

\begin{sk}
	Greedy algorithm.
\end{sk}

\begin{thm}[4-color Theorem, 1970's]
	If $G$ is planar, then $\chi(G) \le 4$.
\end{thm}

\begin{prop}
	If $G$ is planar, then $\chi(G) \le 6$.
\end{prop}

\begin{dem}
	Induction on $n$.

	Since $G$ is planar, $e(G) \le 3n - 6$, thus $\delta(G) \le 5$. Pick $v$ with degree at most $5$. 
	Define $G'$ as $G$ without $v$, then $G'$ has a proper coloring. Now, $v$ has at most five neighbors, thus we can pick one color for $v$ out of six such that no neighbor of $v$ has this color.
\end{dem}

\begin{figure}[ht]
    \centering
	\incfig[.8]{five-color-theorem}
    \caption{Five color theorem}
    \label{fig:second-case-five-color-theorem}
\end{figure}

\begin{thm}
	If $G$ is planar, then $\chi(G) \le 5$.
\end{thm}

\begin{dem}
	Induction on $n$.

	Since $G$ is planar, $e(G) \le 3n - 6$, thus $\delta(G) \le 5$. Pick $v$ with degree at most $5$. 
	Define $G'$ as $G$ without $v$, then $G'$ has a proper coloring. Now, $v$ has at most five neighbors. If there at most four colors are used in the neighbors of $v$, we can paint $v$ with a distinct color.

	Suppose all neighbors of $v$ have different colors. Let's call the neightbors $u_1, u_2, u_3, u_4, u_5$, in clockwise order, with colors $1, 2, 3, 4, 5$.

	Define $\left.G'\right._a^b$ as the subgraph of $G'$ that only contains vertices with colors $a$ and $b$. Let $H_a^b$ be the connected component of $\left.G'\right._a^b$ that contains $u_a$.
	\begin{enumerate}[label = \textbullet]
		\item \textbf{\boldmath If there exists $a, b$ such that $u_b \not\in H_a^b$,} then we flip the colors $a$ and $b$ inside $H_a^b$ and define $c(v) := a$.
		\item \textbf{\boldmath If, for all $a, b$, $u_b \in H_a^b$,} $H_{1, 3}$ and $H_{2, 4}$ are vertex disjoint, but have to go through each other; a contradiction. See \cref{fig:second-case-five-color-theorem}.
	\end{enumerate}
\end{dem}


\begin{thm}[Erd\H{o}s-Stone, 1946]
	\[
		\ex(n, H) = \left(1 - \frac{1}{\chi(H) - 1} + o(1)\right)\binom{n}{2}.
	\]
\end{thm}

\begin{sk}
	The example is the Turán's Graph $T_{\chi(H)-1}(n)$.
\end{sk}
