\lecture{4}{January 07, 2021}{}
\begin{defn}
	Let $r(G, H)$ denote the minimum $n$ such that, for every coloration $c \colon E(K_n) \to \{R, B\}$, there must exist a red $G$ or a blue $H$.
\end{defn}

\begin{prop}
	\[
		\chi(G) \le \Delta(G) + 1.
	\]
\end{prop}

\begin{sk}
	Greedy algorithm.
\end{sk}

\begin{thm}[4-color Theorem, 1970's]
	If $G$ is planar, then $\chi(G) \le 4$.
\end{thm}

\begin{prop}
	If $G$ is planar, then $\chi(G) \le 6$.
\end{prop}

\begin{dem}
	Induction on $n$.

	Since $G$ is planar, $e(G) \le 3n - 6$, thus $\delta(G) \le 5$. Pick $v$ with degree at most $5$. 
	Define $G'$ as $G$ without $v$, then $G'$ has a proper coloring. Now, $v$ has at most five neighbors, thus we can pick one color for $v$ out of six such that no neighbor of $v$ has this color.
\end{dem}

\begin{figure}[ht]
    \centering
	\incfig[.8]{five-color-theorem}
    \caption{Five color theorem}
    \label{fig:second-case-five-color-theorem}
\end{figure}

\begin{thm}
	If $G$ is planar, then $\chi(G) \le 5$.
\end{thm}

\begin{dem}
	Induction on $n$.

	Since $G$ is planar, $e(G) \le 3n - 6$, thus $\delta(G) \le 5$. Pick $v$ with degree at most $5$. 
	Define $G'$ as $G$ without $v$, then $G'$ has a proper coloring. Now, $v$ has at most five neighbors. If there at most four colors are used in the neighbors of $v$, we can paint $v$ with a distinct color.

	Suppose all neighbors of $v$ have different colors. Let's call the neightbors $u_1, u_2, u_3, u_4, u_5$, in clockwise order, with colors $1, 2, 3, 4, 5$.

	Define $\left.G'\right._a^b$ as the subgraph of $G'$ that only contains vertices with colors $a$ and $b$. Let $H_a^b$ be the connected component of $\left.G'\right._a^b$ that contains $u_a$.
	\begin{enumerate}[label = \textbullet]
		\item \textbf{\boldmath If there exists $a, b$ such that $u_b \not\in H_a^b$,} then we flip the colors $a$ and $b$ inside $H_a^b$ and define $c(v) := a$.
		\item \textbf{\boldmath If, for all $a, b$, $u_b \in H_a^b$,} $H_{1, 3}$ and $H_{2, 4}$ are vertex disjoint, but have to go through each other; a contradiction. See \cref{fig:second-case-five-color-theorem}.
	\end{enumerate}
\end{dem}

\begin{lem}
	Se $T$ é uma árvore, então $\chi(T) \le 2$.
\end{lem}
\begin{sk}[1]
	Induction on number of vertices. Paint a leaf by the oposite color of its neighbor.
\end{sk}

\begin{thm}[Erd\H{o}s-Stone, 1946]\label{thm:erdos-stone-1946}
	\[
		\ex(n, H) = \left(1 - \frac{1}{\chi(H) - 1} + o(1)\right)\binom{n}{2}.
	\]
\end{thm}


\begin{sk}
	The example is the Turán's Graph $T_{\chi(H)-1}(n)$.

	Let $r = \chi(H)$. 
	We'll show it by induction on $r$.

	If $r \le 2$, then the theorem says $\ex(n, H) = o(n^2)$, which is true by \hyperref[thm:kovari-sos-turan-1954]{K\H{o}vári-Sós-Turán, 1954}.

	\begin{lem}\label{lem:subgraph-large-density}
		Let $\varepsilon > 0$ such that $\epsilon\binom{n}{2} > \binom{m_0}{2}$ and $G$ be a graph with density $\beta$. Then, there exists $G^* \subset G$ with $m \ge m_0$ vertices and  \[
			\delta(G^*) \ge (\beta - \varepsilon)m.
		\]
	\end{lem}
	\begin{sk}
		Throw away vertices with small degree. The first one we threw away had degree at most $< (\beta - \varepsilon)n$, the second one had degree at most $< (\beta - \varepsilon)(n-1)$, and so on.

		If we threw $n-m_0$ vertices away, then
		\begin{align*}
			e(G) &< (\beta - \varepsilon) (n + (n-1) + \cdots + m_0) + \binom{m_0}{2}\\
				 &< (\beta - \varepsilon) \binom{n}{2} + \binom{m_0}{2}\\
				 &< \beta\binom{n}{2}.
		\end{align*}
	\end{sk}

	The graph $H$ is contained in $K_{r}(t)$, the complete $r$-partite with $t$ vertices on each part, with $t = t(H)$.

	Suppose $e(G) \ge \left(1 - \frac{1}{r-1} + \alpha\right)\binom{n}{2}$. Applying \cref{lem:subgraph-large-density} with $\varepsilon = \frac{\alpha}{2}$, $m_0 = \frac{\alpha n}{2}$, and $\beta = 1 - \frac{1}{r-1} + \alpha$, we conclude that there exists $G^* \subset G$, with $m \ge \frac{\alpha n}{2}$ vertices, and $\delta(G^*) \ge \left(1 - \frac{1}{r-1} + \frac{\alpha}{2}\right)m$.

	Induction hypothesis implies that, for large $n$, $G^*$ has a copy $F$ of $K_{r-1}(q)$, the complete $(r-1)$-partide graph with $q$ vertices on each part, for $q > \frac{2(t-1)}{(r-1)\alpha}$.

	Let $X = V(G^*) \backslash V(F)$.
	Let $Y$ be the set of vertices in $X$ that have at least $(r-2)q + t$ neighbors in $V(F)$.

	Let's call $F_1, F_2, \dots, F_{r-1}$ the parts of $F$, a complete $(r-1)$-partite graph. Let's count the number of \emph{hyper-cherries} $(v, S_1, S_2, \dots, S_{r-1})$, in which $v \in X$, $S_1 \subset F_1$, \dots, $S_{r-1} \subset F_{r-1}$, and $v \sim u$, for all $u$ in some $S_i$. See \cref{fig:erdos-stone-theorem}.

	For each vertex $v$ in $Y$ (of $|Y|$), there are $\prod_{i} \binom{\deg_i(v)}{t} \ge \binom{q}{t}^{r-2}$ hyper-cherries. On the other hand, for each possible subsets $S_1, \dots, S_{r-1}$ (of $\binom{q}{t}^{r-1}$), there are at most $t-1$ hyper-cherries. This implies \[
		|Y| \le (t-1)\binom{q}{t}.
	\]

	Thereore, the number of edges between $X$ and $V(F)$ is at most \[
		\left(m - (r-1)q - \binom{q}{t}(t-1)\right) \left((r-2)q + t - 1\right) + \binom{q}{t}(t-1)(r-1)q,
	\]
	which simplifies to \[
		m ((r-2)q + t - 1) + \text{constant}.
	\]

	On the other hand, since every vertex of $V(F)$ has degree at least $\left(1 - \frac{1}{r-1} + \frac{\alpha}{2}\right)m$ the number of vertices between $X$ and $V(F)$ is at least \[
		\left(\left(1 - \frac{1}{r-1} + \frac{\alpha}{2}\right)m - (r-2)q\right)(r-1)q,
	\]
	which simplifies to \[
		m\left((r-2)q + \frac{(r-1)q\alpha}{2}\right) + \text{constant},
	\]
	which yeilds to a contradiction to large $n$ (i.e. large $m$).
\end{sk}

\begin{figure}[ht]
    \centering
	\incfig[.7]{erdos-stone-theorem}
    \caption{Hyper-cherry}
    \label{fig:erdos-stone-theorem}
\end{figure}
