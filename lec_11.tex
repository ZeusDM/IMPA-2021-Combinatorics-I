\newpage
\section{Janson's Inequality and Aplications}
\lecture{11}{February 09, 2021}{YouTube, Lec. 11}

\begin{lem}[Harris, 1960; FKG, 1970s]\label{lem:harris-1960}
	If $E$ and $F$ are increasing events (or both decreasing), then \[
		\Prob(E \cap F) \ge \Prob(E)\Prob(F).
	\]

	If $E$ is increasing and $F$ is decreasing, then \[
		\Prob(E \cap F) \le \Prob(E)\Prob(F).
	\]
\end{lem}
\begin{dem}
	Left to the reader.
\end{dem}

\begin{thm}[Janson, 1987]\label{thm:janson-1987}
	Let $A_1, \dots, A_m \subset [N]$. Define the events $B_j := \{A_j \subset R\}$, in which $\Prob(i \in R) = p$, $\forall i \in [N]$ independently.

	Let $X = \sum_{j=1}^m \Ind[B_j]$,  $\mu = \Exp[X]$,  $\Delta = \sum_{i \sim j} \Prob(B_i \cap B_j)$, in which  $i \sim j$ means $A_i \cap A_j \neq \varnothing$.

	Then, \[
		\Prob\left( \bigcap_{j=1}^m \overline{B_j}\right) \le
		\begin{cases}
			\exp\left(-\mu + \frac{\Delta}{2}\right)\text{, if }\Delta \le \mu,\\
			\exp\left(-\frac{\mu^2}{2\Delta}\right)\text{, if }\Delta \ge \mu.
		\end{cases}
	\]
\end{thm}
\begin{dem}[of the first bound, by Boppara-Spencer]
	\begin{align*}
		\Prob\left(\bigcap_{j=1}^{m} \overline{B_j}\right) &= \prod_{j=1}^{m} \left( 1 - \Prob\left( B_j \mid \overline{B_1} \cap \cdots \cap \overline{B_{i-1}}\right)\right)\\
		&\le \exp\left(- \sum_{j=1}^m \Prob(B_j \mid \overline{B_1} \cap \cdots \cap \overline{B_{i-1}})\right)\\
		&\signexpl{\le}{\cref{prop:dem:janson-1987}}{-4.5em} \exp\left(-\mu + \frac{\Delta}{2}\right).
	\end{align*}

	It suffices to prove the following claim:

	\begin{prop}\label{prop:dem:janson-1987}
		\[
		\Prob\left(B_j \mid \overline{B_1} \cap \cdots \cap \overline{B_{i-1}}\right) \ge \Prob(B_j) - \sum_{\substack{i\sim j \\ i < j}}\Prob(B_i \cap B_j)
		\]
	\end{prop}
	\begin{dem}
		Let $E = \bigcap_{\substack{i \sim j \\ i < j}}\overline{B_j}$, and $F = \bigcap_{\substack{i\not\sim j \\ i < j}} \overline{B_i}$. Then,
		\begin{align*}
			\Prob\left(B_j \mid E \cap F\right)
			&\ge \Prob(B_j \cap E \mid F) \\
			&= \Prob(B_j \mid F) \Prob(E \mid B_j \cap F)\\
			&= \Prob(B_j)\Prob(E \mid B_j \cap F)\\
			&\signexpl{\ge}{\hyperref[lem:harris-1960]{Harris}}{-1.3em} \Prob(B_j) \Prob(E \mid B_j)\\
			&\signexpl{\ge}{union bound}{-2.6em} \Prob(B_j) \left(1 - \sum_{\substack{i \sim j \\ i < j}}\Prob(B_i \mid B_j)\right)\\
			&\ge \Prob(B_j) - \sum_{\substack{i \sim j \\ i < j}}\Prob(B_i \cap B_j).
		\end{align*}
	\end{dem}
\end{dem}
\begin{dem}[of the second bound]
	Suppose that $\Delta \ge \mu$.

	Let $S$ be a random subset of $[m]$, with $\Prob(j \in S) = q$, $\forall j$, independently.

	Let's calculate the expectated values of $\mu(S)$ and $\Delta(S)$.
	\begin{gather*}
		\Exp[\mu(S)] = \sum \Prob(j\in S) \Prob(B_j) = q\mu.\\
		\Exp[\Delta(S)] = \sum_{i \sim j} \Prob(i, j \in S) \Prob(B_i \cap B_j) = q^2\Delta.
	\end{gather*}

	Thus,
	\begin{align*}
		\Exp\left[\mu(S) - \frac{\Delta}{2}\right] &= qu - \frac{q^2\Delta}{2}\\
		&= \frac{\mu^2}{2\Delta},
	\end{align*}
	for $q := \frac{\mu}{\Delta} \le 1$.

	Thus, there exists  $S$ such that $\mu(S) - \frac{\Delta(S)}{2} \ge \frac{\mu}{2\Delta}$.

	Then,
	\begin{align*}
		\Prob\left(\bigcap_{j=1}^m \overline{B_j}\right)
			&\le \Prob\left(\bigcap_{j \in S} \overline{B_j}\right)\\
			&\le \exp\left(-\mu(S) + \frac{\Delta(S)}{2}\right)\\
	\end{align*}
\end{dem}

\subsection{Triangle-freeness is not sharp}

	We saw before that \[
		\Prob(K_3 \subset G(n, p)) \to
		\begin{cases}
			0\text{, if } p \ll \frac{1}{n},\\
			1\text{, if } p \gg \frac{1}{n}.
		\end{cases}
	\]
	What happens if $p = \frac{c}{n}$?
\begin{lem}	
	If $p = \frac{c}{n}$, then, as $n \to \infty$, \[
		\Prob(K_3 \not\subset G(n, p)) \to \exp\left(-\frac{c^3}{6}\right)
	\]	
\end{lem}

\begin{dem}
	Let $T_1, T_2, \dots, T_{\binom{n}{3}}$ be the triangles in $K_n$. Define the events $B_j := \{T_j \in G(n, p)\}$.

	On one hand,
	\begin{align*}
		\Prob\left(K_3 \not\subset G(n, p)\right)
			&= \Prob\left(\bigcap_{j=1}^{\binom{n}{3}} \overline{B_j}\right) \\
			&= \prod_{i=1}^{\binom{n}{3}} \Prob\left( \overline{B_i} \mid \overline{B_1} \cap \cdots \cap \overline{B_{i-1}}\right) \\
			&\ge \prod_{i=1}^{\binom{n}{3}} \Prob\left(\overline{B_i}\right) \\
			&= (1 - p^3)^{\binom{n}{3}} \to \exp\left(-\frac{c^3}{6}\right).
	\end{align*}

	On the other hand, $\mu = p^3\binom{n}{3}$, and $\Delta \approx p^2 n \mu \ll \mu$. Thus,
	\begin{align*}
		\Prob\left(K_3 \not\subset G(n, p)\right)
			&= \Prob\left(\bigcap_{j=1}^{\binom{n}{3}} \overline{B_j}\right) \\
			&\signexpl{\le}{\hyperref[thm:janson-1987]{Janson}}{-1.5em} \exp\left(-(1+o(1))p^3\binom{n}{3}\right)\\
			&\to \exp\left(-\frac{c^3}{6}\right).
	\end{align*}
\end{dem}

\subsection{Chromatic number of a random graph}

\begin{thm}[Bollobás, 1980s]\label{thm:bollobas-1980}
	\[\chi\left(G\left(n, \tfrac{1}{2}\right)\right) = \big(1 + o(1)\big) \frac{n}{2\log_2{n}},\]
	with high probability.
\end{thm}

\begin{sk}[for upper bound]
	We will start the proof with the following lemma:

	\begin{lem}\label{lem:thm:bollobas-1980}
		For all subsets $S \subset V\left(G\left(n, \tfrac{1}{2}\right)\right)$, $\left|S\right| \ge \frac{n}{(\log n)^2}$, there exists an independent set $I \subset S$, with size $\ge (2-\varepsilon)\log_2n$.
	\end{lem}
	\begin{sk}
		Let $N = \binom{|S|}{2}$; $m = \binom{|S|}{k}$; $A_1, \dots, A_m$ be sets of edges of the complete bipartite graphs inside with vertices on $S$; $B_1, \dots, B_m$ be the events $\{A_j \subset R\}$; and $R = \overline{G(n, \tfrac{1}{2})}[S]$.

		Let's set things up for applying \nameref{thm:janson-1987}.
		\begin{gather*}
			\mu = \binom{|S|}{k}2^{-\binom{k}{2}} \ge \left(\frac{|S|}{k} 2^{-k/2}\right)^k \ge \left(\frac{n^{\varepsilon/2}}{(\log n)^3}\right)^k \ge n^{(\varepsilon\log_2n)/2}\\
			\delta \signexpl{\lesssim}{worst cases are $\ell = 2$ or $\ell = k - 1$}{-10em} \mu^2\frac{(\log n)^8}{n^2} + \mu \frac{1}{n^{1-\varepsilon}}.  
		\end{gather*}
	\end{sk}

	From \cref{lem:thm:bollobas-1980}, we some have disjoint independent sets with size at least $(2-\varepsilon)\log_2n$ elements, and at most $\frac{n}{(\log n)^2}$ outside a independent set. Thus,  \[
		\chi\left(G(n, \tfrac{1}{2})\right) \le \frac{n}{(2-\varepsilon)\log_2n} + \frac{n}{(\log{n})^2}.
	\]

\end{sk}
