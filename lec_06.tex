\newpage\section{Extremal olympiad-like problems}
\lecture{6}{January 19, 2021}{YouTube, Lec. 7}

\begin{defn}
	$\mathcal{A} \subset \mathcal{P}([n])$ is an \emph{anti-chain} se  $A \not\subset B$, para todo $A, B \in \mathcal{A}$, $A \neq B$.
\end{defn}

\begin{thm}[Sperner, 1910's] \label{thm:sperner}
	$\mathcal A \subset \mathcal P([n])$ é uma anti-cadeia $\implies |\mathcal A| \le \binom{n}{n/2}$.
\end{thm}

The example is $\binom{[n]}{n/2}$.

\begin{lem}[LYMB, 1960's]\label{lem:lymb}
	$\mathcal A \subset \mathcal P([n])$ é uma anti-cadeia $\sum_{A \in \mathcal A} \frac{1}{\binom{n}{|A|}} \leq 1$.
\end{lem}

\begin{dem}[of \nameref{thm:sperner}]
	We know that $\binom{n}{k} \le \binom{n}{n/2}$. Thus, by \nameref{lem:lymb}, \[
		1 \ge \sum_{A\in \mathcal{A}} \frac{1}{\binom{n}{|A|}} \ge \sum_{A\in \mathcal{A}} \frac{1}{\binom{n}{n/2}} = \frac{|\mathcal{A}|}{\binom{n}{n/2}}
	\]
\end{dem}

\begin{dem}[of \nameref{lem:lymb}]
	Let's count the pairs $(\pi, A)$ such that $\pi$ is a permutation of $[n]$, $A \in \mathcal{A}$, and  $\{\pi(1), \pi(2), \dots, \pi(|A|)\} = A$.

	For each $A \in \mathcal{A}$, the number of $\pi$ such that $\{\pi(1), \pi(2), \dots, \pi(|A|)\}$ is equal to $|A|!(n - |A|)!$.

	For each $\pi$, the number of $A \in \mathcal{A}$ such that $\{\pi(1), \dots, \pi(|A|)\}$ is at most $1$, since $\mathcal{A}$ is an anti-chain.

	Therefore, \[
		\sum_{A \in \mathcal{A}} |A|!(n-|A|)! \le n!
		\implies
		\sum_{A \in \mathcal{A}} \frac{1}{\binom{n}{|A|}} \le 1.
	\]
\end{dem}

\begin{defn}
	$\mathcal{A}$ is \emph{intersecting} if $A \cap B \neq \varnothing$, for all  $A, B \in \mathcal{A}$.
\end{defn}

\begin{prop}
	$\mathcal{A} \subset \mathcal{P}([n])$ is intersecting $\implies |A| \le 2^{n-1}$.
\end{prop}

\begin{sk}
	At most one of $(S, \overline{S})$ can be on $\mathcal{A}$.
\end{sk}

\begin{thm}[Erd\H{o}s-Ko-Rado, 1961]
	$\mathcal{A} \subset \binom{[n]}{k}$ is intersecting $\implies |A| \le \binom{n-1}{k-1}$, for $k < \frac{n+1}{2}$.
\end{thm}

\begin{dem}
	Let's count the number of pairs $(\pi, A)$ such that $\pi$ is a circular permutation and $A \in \mathcal{A}$ is an interval in $\pi$.

	For each $A \in \mathcal{A}$, the number of permutations such that $A$ is an interval in $\pi$ is $k!(n-k)!$

	For each circular permutation $\pi$, the number of $A \in \mathcal{A}$ such that $A$ is an interval in $\pi$ is at most $k$.

	Therefore, \[
		|A|k!(n-k)! \le (n-1)!k
		\implies
		|A| \le \binom{n-1}{k-1}.
	\]
\end{dem}
