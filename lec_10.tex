\subsection{Connectivity}
\lecture{10}{January 28, 2021}{YouTube, Lec. 9}

\begin{thm}
	If $p < (1-\varepsilon)\frac{\log{n}}{n}$, then $G(n, p)$ is not connected with high probability.

	If $p > (1+\varepsilon)\frac{\log{n}}{n}$, then $G(n, p)$ is connected with high probability.
\end{thm}
\begin{dem}[of the first part]
	Directly from \cref{prop:isolatedvertex}.
\end{dem}
\begin{dem}[of the second part]
	A graph $G$ is disconnected if, and only if, there exits a complete bipartite graph which is a subgraph of $\overline G$. 

	For $k \in \{1, \dots, n/2\}$, let $X_k$ be the number of copies of $K_{k, n-k}$ in $\overline{G(n,p)}$.
	\begin{align*}
		\Exp[X_k] &= \binom{n}{k}(1-p)^{k(n-k)} \\
				  &\le \left(\frac{en}{k}e^{-p(n-k)}\right)^k \\
				  &\le \left(\frac{en}{k}n^{-\left(1+\varepsilon\right)\left(1-\frac{k}{n}\right)}\right)^k\\
				  &\le n^{-\varepsilon k /2} \to 0
	\end{align*}

	Since $X_k = 0$, for $k \in \{1, \dots, n/2\}$ with high probability, then $G(n,p)$ is connected with high probability.
\end{dem}

\subsection{Thresholds for General Properties}

\begin{defn}[Sharp threshold]
	An event $\mathcal A = \mathcal A(n)$ has a \emph{sharp threshold} if there exists $p_c$ such that:
	\begin{enumerate}[label = \textbullet] 
		\item $p \ge (1+\varepsilon)p_c \implies \Prob(\mathcal A) \to 1$, as  $n \to \infty$;
		\item $p \ge (1-\varepsilon)p_c \implies \Prob(\mathcal A) \to 0$, as  $n \to \infty$.
	\end{enumerate}
\end{defn}

\begin{defn}[Coarse threshold]
	An event $\mathcal A = \mathcal A(n)$ has a \emph{coarse threshold} if there exists $p_c$ such that:
	\begin{enumerate}[label = \textbullet] 
		\item $p \gg p_c \implies \Prob(\mathcal A) \to 1$, as  $n \to \infty$;
		\item $p \ll p_c \implies \Prob(\mathcal A) \to 0$, as  $n \to \infty$.
	\end{enumerate}
\end{defn}

\begin{thm}[Bollobás--Thomason, 1980s]
	Every increasing property (in the sense of adding edges) has a coarse threshold.
\end{thm}

\begin{sk}
	Define $p_\varepsilon = \inf \{p : \Prob(G(n, p) \in \mathcal A) \ge \varepsilon\}$.

	\begin{prop}\label{prop:bt-1980}
		 There exists $C = C(\varepsilon)$ such that $\Prob(G(n, Cp_\varepsilon) \in \mathcal A) \ge 1-\varepsilon$.
	\end{prop}
	\begin{sk}
		We shall use sprinkling.

		$G_1, \dots, G_C$ indepentent copies of $G(n, p_\varepsilon)$. \[
			G_1 \cup \cdots \cup G_c \sim G\left(n, 1-(1-p_\varepsilon)^c\right) \subset G(n, Cp_\varepsilon).
		\]

		Then, \begin{align*}
			\Prob(G(n, Cp_\epsilon) \in \mathcal A) &\ge \Prob\left( \bigcup_{i=1}^C G_i \in \mathcal A \right)\\
			&\ge 1 - (1 - \epsilon)^C\\
			&\ge 1 - e^{-C\epsilon}\\
			&\ge 1 - \epsilon,
		\end{align*}
		if $C \ge \frac{-\log(\epsilon)}{\epsilon}$.
	\end{sk}

	Recall that $p \gg p_c \implies \forall \varepsilon > 0, p > C(\varepsilon)p_c$; and $p \ll p_c \implies \forall \varepsilon > 0, p < \frac{p_c}{C(\varepsilon)}$.
\end{sk}

\subsection{Hamiltonian Cycles}
\begin{thm}[Dirac]
	If $\delta(G) \ge n/2$, then there exists a Hamiltonian cycle.
\end{thm}
\begin{sk}
	Take the longest path $v_1, v_2, \dots, v_k$. Clearly, $k > n/2$. Use pidgeonhole principle to find $v_1 \sim v_{x+1}$ and  $v_k \sim v_{x}$. We have a cycle $v_1, v_2, \dots, v_x, v_k, v_{k-1}, \dots, v_{x+1}$.

	If there is a vertex outside of this cycle, it must connect to some vertex inside the cycle. If such thing happens, we can find a larger path than the one we started with; a contradiction.

	Therefore, this cycle goes through all vertices.
\end{sk}

\begin{thm}
	In the random graph process $G_m$,  \[
		\min\{m : \delta(G_m) \ge 2\} = \min\{m : G_m \text{\ has a Hamiltonian cycle}\}
	\]
	with high probability.
\end{thm}

