\documentclass[a4paper, 10pt]{article}
\usepackage[utf8]{inputenc}
\usepackage[english]{babel}
%\usepackage{etoolbox}
\usepackage[symbol]{footmisc}
%\usepackage{lmodern}
\usepackage[margin=2.5cm, marginpar = 2cm]{geometry}
\usepackage{hyperref}
\usepackage[section, boxed, prob-boxed, dem-boxed]{zeus}
\usepackage{mathtools}
\usepackage[only, mapsfrom]{stmaryrd}
\usepackage{parskip}
\usepackage{soul}
\usepackage{todonotes}
\setlist{nosep, noitemsep, label = (\textit{\roman*})}

\title{\sffamily \bfseries Combinatorics I Lecture Notes}
\author{\sc Guilherme Zeus Dantas e Moura\\\href{mailto:zeusdanmou@gmail.com}{\texttt{zeusdanmou@gmail.com}}}
\date{IMPA\\[3pt] January -- February 2021\\[3pt] Last update: \today}

\newcommand{\lecture}[3]{
	%\newpage
	%\section{#3 (#2)}
	\todo[bordercolor = gray!30!white, backgroundcolor = gray!30!white, noline]{\scalebox{.65}{\sf #3}\\[-1.5mm]\scalebox{.65}{\sf #2}}
}

\usepackage{import}
\usepackage{pdfpages}
\usepackage{transparent}
\usepackage{xcolor}

\newcommand{\incfig}[2][1]{%
    \def\svgwidth{#1\columnwidth}
    \import{./figures/}{#2.pdf_tex}
}

\newcommand{\correct}[2]{\textcolor{Red!90!black}{\st{#1}} \textcolor{ForestGreen}{#2}}

\pdfsuppresswarningpagegroup=1

\DeclareMathOperator{\ex}{ex}
\DeclareMathOperator{\Var}{Var}

\begin{document}
    \maketitle
	\sloppy
	
		This is IMPA's master class Combinatorics 1, instructed by Robert Morris, with the help of Letícia Mattos.
		All errors are my responsability.

		Use these notes only as a guide. There is a non-trivial chance that some things here are wrong or incomplete (especially proofs).

		This class is being taught remotely via Gooogle Meet and \href{https://www.youtube.com/playlist?list=PLo4jXE-LdDTSkmHd3xNGhcObfWXvpwmCL}{YouTube videos}. The recommended material can be found \href{https://impa.br/wp-content/uploads/2017/04/28CBM_04.pdf}{here}.

		\hfill

		\begin{figure}[h]
			\centering
			\incfig[.8]{testing-one}
			\caption{This is a graph.}
			\label{fig:testing-one}
		\end{figure}

		\hfill

	\newpage
	\tableofcontents

	\newpage
	% start lectures
    \lecture{1}{January 04, 2021}{}

\section{Which problems we'll study?}

In this summer course, we'll study extremal, counting and probabilistic problems. Here are some examples:

\begin{prob}
	Let $A$ be a subset of $\{1, 2, \dots, 2n\}$ such that $a \nmid b$, for all $a \neq b \in A$.

	How large can $|A|$ be?
\end{prob}

\begin{sol}
	$A = \{n+1, \dots, 2n\}$ is a good example. This yields $|A| = n$.

	Consider the partition of $\{1, 2, \dots, 2n\}$ given by the following sets:
	\begin{enumerate}[label = \textbullet]
		\item $\{2^t\}$
		\item $\{3 \cdot 2^t\}$
		\item $\{5 \cdot 2^t\}$
		\item[$\vdots$]
		\item $\{(2n-1) \cdot 2^t\}$
	\end{enumerate}

	There can't be two elements in the same set of the partition, so $|A| \le n$. 
\end{sol}

\begin{prob}
	Let $A$ be a subset of $\{1, 2, \dots, 2n\}$ such that $a + b \neq c$, for all $a, b, c \in A$. We'll call such set \emph{sum-free}.

	How large can $|A|$ be?
\end{prob}

\begin{sol}	
	$A = \{n+1, \dots, 2n\}$ is a good example. Another good example are the odd numbers. Both yield $|A| = n$.

	Suppose $|A| \ge n+1$. Let $a = \max A$.

	Consider the following partition with $\floor{\frac{a}{2}}$ sets:
	\begin{enumerate}[label = \textbullet]
		\item $\{1, a-1\}$
		\item $\{2, a-2\}$
		\item[$\vdots$]
		\item $\{\floor{\frac{a}{2}}, \ceil{\frac{a}{2}}\}$
	\end{enumerate}

	There can't be two elements in the same set of the partition.

	If $a \le 2n-1$, then there are at most $n - 1$ sets listed above, which implies $|A| \le n$.

	If $a = 2n$, then $n \not\in A$, and then the $n-1$ first sets listed above cover $A$, thus $|A| \le n$.
\end{sol}

\begin{thm}[Schur, 1916] \label{thm:schur}
	Given $c \colon \ZZ_{>0} \to \{1, \dots, r\}$, the there are $x, y, z$ such that:
	\begin{enumerate}[label = \textbullet]
		\item $x + y = z$
		\item $c(x) = c(y) = c(z)$
	\end{enumerate}
\end{thm}

\begin{prob}
	How many sum-free sets are in $[n]$?
\end{prob}

\begin{conj}[Cameron and Erd\H{o}s]
	The number of sum-free sets in $[n]$ is $\le C\cdot2^{n/2}$.
\end{conj}

\newpage
\section{Ramsey's Theory}

\begin{thm}[Ramsey's Theorem]\label{thm:ramsey}
	If $c: \binom{\NN}{2} \to \{1, \dots, r\}$, then there exists $A \subset \NN$ infinite and monochromactic, i.e, such that $c(ab) = c$, for all $a, b \in A$.
\end{thm}

\begin{dem}[of \cref{thm:ramsey}]
	Let $S_0 = \NN$.

	For each $i$, do the following:
	Pick $v_i \in S_{i-1}$. Look at the colors of $\{v_i, u\}$, for $u$ in $S_{i-1}$. Since  $S_{i-1}$ is infinite and there are finitely many colors, there is some color that appears infinitely many times; we'll call this color $c_i$, and define $S_i = \{u \in S_{i-1} \colon c(\{v_i, u\}) = c_i\}$.

	Now, we have an infinite sequence $v_1, v_2, \dots$, such that $c(\{v_i, v_j\}) = c_i$, for $i < j$. Since there are finitely many colors, there is some color that appears in infinitely many $c_i$'s; call this color $c$, and define $A = \{v_i \colon c_i = c\}$.

	The set $A$ satisfies our condition.
\end{dem}

\begin{dem}[of \cref{thm:schur}]
Given a coloring $c: \NN \to \{1, \dots, r\}$, we define $c': \binom{\NN}{2} \to \{1, \dots, r\}$ by $c'(\{a, b\}) = c(b - a)$, for $b > a$. 

By \cref{thm:ramsey}, there is $A$ infinite and monochromactic. Pick $x < y < z \in A$, then we have $c(y - x) = c(z - y) = c(z - x)$, and $(y - x) + (z - y) = z - y$, so we're done!
\end{dem}

\begin{defn}[Ramsey Number]
	Let $R(k)$ denote the smallest $n$ such that, for every coloring with two colors of the edges of the complete graph $K_n$, i.e., for every $c: E(K_n) \to \{R, B\}$, there exists a monochromatic copy of $K_k$.

	Let $R(s, t)$ denote the smallest $n$ such that, for every coloring with two colors of the edges of the complete graph $K_n$, i.e., for every $c: E(K_n) \to \{R, B\}$, there exists a red copy of $K_s$ or a blue copy of $K_t$.

	Clearly, $R(k) = R(k, k)$.
\end{defn}

\begin{thm}
	\[
		R(k) \lesssim 2^{2k}.
	\]
\end{thm}

\begin{sk}	
	Let $n = 2^{2k}$, and pick any coloring $c$ of $K_{n}$.
	Let $S_0 = [n]$.

	For each $i < 2k$ do the following:
	Pick $v_i \in S_{i-1}$. Look at the colors of $\{v_i, u\}$, for $u$ in $S_{i-1}$. There is some color that appears more times; we'll call this color $c_i$, and define $S_i = \{u \in S_{i-1} \colon c(\{v_i, u\}) = c_i\}$.

	\correct{Note that, the size of $S_m$ is at least $\frac{n}{2^m} \ge 1$.}{This is not quite correct. At each step, we're taking one vertice away, and then dividing by two.}

	Now, we have an sequence $v_1, v_2, \dots, v_{2k-1}$, such that $c(\{v_i, v_j\}) = c_i$, for $i < j$. Since there are two colors, there is some color that appears at least $k$ times; call this color $c$, and define $A = \{v_i \colon c_i = c\}$. The size of $A$ is at least $k$. Pick any subset $B$ of $A$ that has exactly $k$ elements.

	The subgraph of $K$ given by deleting all vertices but those in $B$ is a monochromatic copy of $K_k$.
\end{sk}

\begin{lem}\label{lem:upperrecurenceramsey}
	\[
		R(s, t) \le R(s-t, t) + R(s, t-1).
	\]
\end{lem}
\begin{dem}
	Let $n = R(s, t) - 1$. By definition, there exists a coloring $c \colon E(K_n) \to \{R, B\}$ without a red $K_s$ or a blue $K_t$.

	Pick any vertex $v$. $v$ it connected to some of the vertices through a red edge, which we'll put in the set $S_R$; the others are connected to $v$ through a blue edge, those we'll put in the set $S_B$.

	Since there are no red $K_s$ or blue $K_t$, there can't be any red $K_{s-1}$ or blue  $K_t$ in $S_R$; thus, $|S_R| \le R(s - 1, t)$. Analougously, $|S_B| \le R(s, t-1)$.

	Therefore,
	\begin{align*}
		R(s, t) - 1 &\le R(s - 1, t) - 1 + R(s, t-1) - 1 + 1\\
		R(s, t) &\le R(s - 1, t) + R(s, t-1).
	\end{align*}
\end{dem}	

\begin{figure}[ht]
    \centering
	\incfig[.7]{red-and-blue-neighbors}
    \caption{$S_R$ and $S_B$.}
    \label{fig:red-and-blue-neighbors}
\end{figure}

\begin{thm}
	\[
		R(s, t) \le \binom{s + t}{s}.
	\]	
\end{thm}

\begin{dem}
	Follows from \cref{lem:upperrecurenceramsey}.
\end{dem}

\begin{thm}[Erd\H{o}s-Szekeres, 1935]
	\[
		R(k) \le \binom{2k}{k} \approx \frac{1}{\sqrt{k}}4^k.
	\]
\end{thm}

    \lecture{2}{January 05, 2021}{YouTube, Lec. 2}

Let's now try to find a lower bound.
It is very difficult to show a good construction.
Luckly, we are not going to do that.

\begin{thm}[Erd\H{o}s, 1947]
	\[
		\left.\sqrt{2}\right.^k \le R(k)
	\]
\end{thm}

\begin{dem}
	Let $n \le \left.\sqrt{2}\right.^k$.
	Let's choose colors randomly. Let \[
		\Prob(c(e) = R) = \frac{1}{2},
	\]
	for every edge $e$ in $K_n$, independently.

	We want to show that \[
		\Prob(\text{there is not a monochromatic copy of $K_k$}) > 0.
	\]

	Let $X$ be the number of monochromactic copies of $K_k$ in $c$. Then, 
	\begin{align*}
		\Exp[X] &= \Exp\left[\sum_{\substack{S \subset K_n\\S\ \text{is a copy of}\ K_k}} \Ind[S\ \text{is monochromatic}]\right]\\
			    &= \sum_{\substack{S \subset K_n\\S\ \text{is a copy of}\ K_k}} \Exp\left[\Ind[S\ \text{is monochromatic}]\right]\\
				&= \sum_{\substack{S \subset K_n\\S\ \text{is a copy of}\ K_k}} \Prob(S\ \text{is monochromatic}))\\
				&= \sum_{\substack{S \subset K_n\\S\ \text{is a copy of}\ K_k}} \left(\frac{1}{2}\right)^{\binom{k}{2} - 1}\\
				&= \binom{k}{n} \left(\frac{1}{2}\right)^{\binom{k}{2} - 1}\\
				&\le 2 \left(\frac{en}{k}\right)^k \left(\frac{1}{2}\right)^{\frac{k(k-1)}{2}}\\
				&\le 2 \left(\frac{e\sqrt{2}}{k}\right)^k \\
				&< 1\text{, for $k \ge 5$.}
	\end{align*}

	Therefore, since $\Exp[X] < 1$, we have $\PP(X = 0) > 0$.
\end{dem}

The bounds have not improved much since then 

\begin{thm}[Conlon, 2009]
	\[
		R(k) \le \frac{4^k}{k^{\sqrt{\log k }}}
	\]
\end{thm}

\newpage
\section{Extremal Graph Theory}
\subsection{Complete Graphs}

\begin{defn}
	Let $\ex(n, H)$ be the maximum number of edges a graph  $G \subset K_n$ can have such that there are no copies of $H$ in $G$.
\end{defn}

\begin{thm}[Mantel, 1907]
	\[
		\ex(n, K_3) = \floor{\frac{n^2}{4}}.
	\]
\end{thm}

\begin{dem}
	The example is the bipartite graph with $\floor{\frac{n}{2}}$ and $\ceil{\frac{n}{2}}$ vertices.

	Let's prove by indution on $n$.

	Now, suppose $G$ does not have a triangle.
	Pick an edge $uv$. Let $G'$ be the graph $G$ deleting $u$ and $v$. The subgraph $G'$ also does not contain triangles, so $e(G') \ge \floor{\frac{n^2}{4}}$.

	Notice that cannot exist $w \in G'$ such that $uw$ and $vw$ are edges of $G$, because $G$ does not have triangles. Therefore, there can be at most $n-2$ edges from $u$ or $v$ to vertices on $G'$. Including the edge $uv$, we conclude that
	\begin{align*}
		e(G) &\le e(G') + n - 1\\
			 &\le \floor{\frac{(n-2)^2}{4}} + n - 1\\
			 &\le \floor{\frac{n^2}{4}}.
	\end{align*}
\end{dem}

\begin{figure}[ht]
    \centering
	\incfig[.6]{edge-uv-on-a-triangle-less-graph}
    \caption{Edge $uv$ on a triangle-free graph.}
    \label{fig:edge-uv-on-a-triangle-less-graph}
\end{figure}

\begin{defn}[Turán's Graph]
	The graph $T_r(n)$ consists of $r$ sets with roughly $n/r$ elements each (some rounded up, some rounded down).; we create an edge $uv$ if, and only if, $u$ and $v$ are on different sets.

	We'll denote by $t_r(n)$ the number of edges in $T_r(n)$.
\end{defn}

\begin{thm}[Turán, 1941]
	\[
		\ex(n, K_{r+1}) = t_r(n) \approx \left(1 - \frac{1}{r}\right)\binom{n}{2}.
	\]
\end{thm}

\begin{dem}
	We'll use induction on $n$. For $n \le r$, we're good.

	Pick a maximal graph $G$ that doesn't have a copy of $K_{r+1}$. Pick a copy of $K_r$, let's call it $H$. Define $G' = G - H$. Of course, $G'$ has no copies of $K_r$; thus $e(G') \le t_r(n-r)$, by induction.

	Futhermore, if $v \in G'$, there can be at most $r - 1$ edges connecting $v$ to some vertex in $H$.

	Wrapping everything up, we have
	\begin{align*}
		e(G) &\le e(G') + (n-r)(r-1) + \binom{r}{2}\\
			 &\le t_r(n-r) + (n-r)(r-1) + \binom{r}{2}\\
			 &\le t_r(n).
	\end{align*}
\end{dem}

\begin{figure}[h]
    \centering
	\incfig[.5]{grafo-de-turan}
    \caption{Turán's Graph}
    \label{fig:grafo-de-turan}
\end{figure}

\subsection{Bipartite Graphs}

\begin{thm}[Erd\H{o}s, 1935]
	\[
		\ex(n, C_4) \le \frac{n^{3/2}}{2}.
	\]
\end{thm}

\begin{dem}
	Let's count cherries! A \emph{cherry} is a pair $(v, \{u, w\})$, in which $vu$ and $vw$ are edges of the graph. 

	Since there is no $C_4$, there is at most one cherry for each pair $\{u, w\}$. This implies that:
	 \begin{align*}
		 \binom{n}{2} \ge \#(\text{cherries}) &= \sum_{v \in G} \binom{d(v)}{2}\\
			&\ge n\binom{\frac{2e(G)}{n}}{2}.
	\end{align*}

	Solving this quadractic inequation on $e(G)$ yields to  \[
		e(g) \ge \frac{n^{3/2}}{2}.
	\]
\end{dem}

\begin{ques}
	For which graphs we have \[
		\ex(n, H) = \Theta(n^2)?
	\]
\end{ques}

\begin{prop}
	For every non-bipartite graph $H$, we have \[
		\ex(n, H) \ge \frac{n^2}{4}.
	\]
\end{prop}

\begin{dem}
	Take $G$ as the complete bipartite graph with $n$ vertices. It has roughly $\frac{n^2}{4}$ vertices and it cannot contain a non-bipartite graph.
\end{dem}

\begin{thm}[Kővári–Sós–Turán, 1954]\label{thm:kovari-sos-turan-1954}
	Let $H$ be a bipartite graph. Then, \[
		\ex(n, H) = o(n^2).
	\]
\end{thm}

\begin{dem}
	Since $H$ is bipartite, there is some $K_{s,t}$ such that $H \subset K_{s, t}$. Then, \[
		\ex(n, H) \le \ex(n, K_{s, t}).
	\]

	Let's bound $\ex(n, K_{s, t})$.

	We'll count $s$-cherries: $(v, S)$, in which $S$ has size $s$ and $vx \in E(G)$ for all $x \in S$.


	There are at most $t-1$ $s$-cherries for each subset $S$ with size $s$. This implies that:
	 \begin{align*}
		 (t-1)\binom{n}{s} \ge \#(\text{cherries}) &= \sum_{v \in G} \binom{d(v)}{s}\\
			&\ge n\binom{\frac{2e(G)}{n}}{s} \ge \frac{e(G)^s}{s^s\cdot n^{s-1}}.
	\end{align*}

	This implies that, for some constant $C$, \[
		e(G) \le C \cdot n^{2 - \frac{1}{s}}
	\]

\end{dem}

\begin{ques}
	For which $H$ it holds that \[
		\ex(n, H) = O(n)?
	\]
\end{ques}

    \subsection{Trees}
\lecture{3}{January 07, 2021}{}

\begin{defn}[Tree]
	A tree is a connected graph that has no cycles.
\end{defn}

\begin{prop}
	Given a graph $G$, the following are equivalent:
	\begin{enumerate}
		\item $G$ is a tree;
		\item $G$ is a maximal graph without cycles, i.e., $G$ does not have cycles and there is no graph $H \supset G$ such that $H$ does not have cycles; 
		\item $G$ is a minimal connected graph, i.e., $G$ is connected and there is no graph $H \subset G$ such that $H$ is connected.
	\end{enumerate}
\end{prop}

\begin{thm}
	Let $T$ be a graph with $k$ vertices. Then, \[
		\frac{(k-2)}{2}n \le \ex(n, T) \le (k - 1) \cdot n.
	\]
\end{thm}

\begin{dem}[of the lower bound]
	Pick $\frac{n}{k-1}$ disjoint $k-1$-cliques. There cannot be a copy of a connected graph with $k$ vertices inside this graph. This graph has roughly \[
		\binom{k-1}{2}\frac{n}{k-1} = \frac{k-2}{2} n
	\] edges.
\end{dem}\

\begin{dem}[of the upper bound] Let's start with a lemma.
	\begin{lem}
		Let $G$ be a graph with mean degree $d$, then, there exists a subgraph $G' \subset G$ with minimum degree at least $\frac{d}{2}$.
	\end{lem}
	\begin{dem}
		While there are vertices with degree smaller than $\frac{d}{2}$, throw them away.

		If we stopped before throwing away all vertices, we're done. Suppose we threw away all vertices. At each step, we threw away at most $\frac{d}{2}$ edges. Since we threw away all edges, this means $n \cdot \frac{d}{2} < e(G) = n\frac{d}{2}$; a contradiction.
	\end{dem}

	\begin{lem}
		Let $G$ be a graph with $\delta(G) \ge k - 1$. Then, there is a copy of $T$ in $G$ for every tree $T$ with $k$ vertices.
	\end{lem}

	\begin{dem}
		We'll use induction on $k$. If $k = 1$, we're done!

		Pick a leaf $v$ of $T$. Its unique edge connects it to $u$. Let $T'$ be the tree without $v$. By induction, there is a copy $C_{T'}$ of $T'$ in $G$. Let $c_u$ be the copy of  $u$ in $C_{T'}$. Since $\deg(c_u) \le k - 2$ in $C_{T'}$, but $\deg(c_u) \ge k-1$ in $G$, there is some vertex that is connected to $u$ outside of $C_{T'}$, say $c_v$. Thus, let $C_{T}$ be $C_{T'}$, adding $c_v$. $C_{T}$ is a copy of $T$ inside $G$.
	\end{dem}

	Finally, $e(G) = (k-1)n \implies \bar{d}(G) = 2(k-1) \implies$ there exists a subgraph $G' \subset G$ such that $\delta(G') \ge k - 1 \implies T \subset G'$.
\end{dem}

\begin{conj}[Erd\H{o}s-Sós, 1960's]
	\[
		\ex(n, T) \le \frac{(k-2)n}{2}
	\]
\end{conj}

\begin{defn}[Random graph of Erd\H{o}s-Rónyi]
	We define $G(n, p)$ as a random distribution of graphs with $n$ vertices, with \[
		\Prob(e \in E(G(n, p))) = p,
	\] chosen independently.
\end{defn}

\begin{lem}[Markov's inequality]
	\[
		\Prob(X \ge t) \ge \frac{\Exp[X]}{t}.
	\]
\end{lem}
\begin{dem}
	Left to the reader. Use the definition of $\Exp[X]$.
\end{dem}

\begin{thm}
	\[
		\ex(n, C_t) \ge O\left(n^{1+\frac{1}{2k-1}}\right) \gg n.
	\]
\end{thm}

\begin{dem}
	Let $t = 2k$. We want to choose $p = p(n)$ such that:
	\begin{enumerate}[label = \textbullet]
		\item $e(G(n, p)) \gg n$;
		\item $C_{2k} \not\subset G(n, p)$.
	\end{enumerate}

	\[
		\Exp[e(G(n, p))] = p\binom{n}{2}.
	\]

	Moreover, $e(G(n, p))$ is a binomial distribution, therefore, $e(G(n, p)) \approx np^2$ with high probability. Thus, we should pick $p \gg 1/n$, i.e., $pn \to \infty$. 

	Define $X$ as the number of copies of $C_{2k}$ in $G(n, p)$.

	\begin{align*}
		\Exp[X] &= \sum_{\substack{\text{copies $S$ of}\\C_{2k}\text{ in }K_n}} \Prob(S \subset G(n, p))\\
				&\approx n^{2k} p^{2k} = (pn)^{2k}.
	\end{align*}

	Let $0 < \varepsilon < \frac{1}{2k-1}$, and define $p = p(n) = n^{-1+\varepsilon}$. Then, we have $pn \gg n^{-1}$ and $(pn)^{2k} \ll pn^2$. Therefore, each of the following happen with high probability:
	\begin{enumerate}[label = \textbullet]
		\item $e(G(n, p)) \approx pn^2$;
		\item The number of copies of $C_{2k}$ in $G(n, p) \approx (pn)^{2k}$.
	\end{enumerate}

	Therefore, the intersection also occours with high probability. Pick a graph $G$ in the intersection.

	For each of the $(pn)^{2k}$ cycles in $G$ delete an edge in it; call this new graph $G'$. Thus $e(G') \approx pn^2 - (pn)^{2k} \approx n^{1+\epsilon} $, and $G'$ has no $C_{2k}$.
\end{dem}

\begin{thm}
	\[
		\ex(n, H) = O(n) \iff H\ \text{does not have cycles}.
	\]
\end{thm}

\begin{dem}
	All the work has been done. The proof, which is simply a jigsaw puzzle, is left to the reader.
\end{dem}

\newpage
\section{Planar graphs}

\begin{defn}[Planar Graph]
	A planar graph is a graph that can be drawn on the plane without having crossing edges. Edges may not be straight.
\end{defn}

\begin{lem}[$V + F - E = 2$]
	Let $G$ be a planar connected graph, and $v(G) \ge 1$. For any planar drawing of $G$, we have \[
		v(G) + f(G) - e(G) = 2.
	\]
\end{lem}

\begin{sk}
	Induction on $e(G)$.

	\begin{enumerate}
		\item \textbf{If there is a leaf,} then we can take it away.
			\begin{align*}
				v(G') &= v(G) - 1, \\
				e(G') &= e(G) - 1, \\
				f(G') &= f(G).
			\end{align*}
		\item \textbf{If there is no leaf,} there is a cycle, take away an edge from the cycle.
			\begin{align*}
				v(G') &= v(G),     \\
				e(G') &= e(G) - 1, \\
				f(G') &= f(G) - 1. 
			\end{align*}
	\end{enumerate}
\end{sk}

	Watch an animated version of this classic demonstration at \href{https://youtu.be/VvCytJvd4H0?t=382}{3Blue1Brown}.

\begin{thm}
	Let $G$ be a planar graph with $n \ge 3$ vertices. Then, \[
		e(G) \le 3n - 6
	\]
\end{thm}

\begin{dem}
	Without loss of generalitym $G$ is maximal.

	Maximal and $n \ge 3$ implies all regions are triangles. Double counting implies \[
		3f(G) = 2e(G).
	\]
	Also, \[
		v(G) + f(G) - e(G) = 2.
	\]
	It follows that	\[
		e(G) = 3n - 6.
	\]
\end{dem}

\begin{thm}
	$K_5$ is not planar.
\end{thm}
\begin{dem}
	\[
		e(K_5) = 10 > 3\cdot5 - 6 = 3v(K_5) - 6.
	\]
\end{dem}

\begin{thm}
	Let $G$ be a triangle-free planar graph with $n \ge 4$ vertices. Then, \[
		e(G) \le 3n - 6
	\]
\end{thm}

\begin{dem}
	Without loss of generalitym $G$ is maximal.

	Maximal and $n \ge 4$ implies all regions have at least $4$ sides. Double counting implies \[
		4f(G) \le 2e(G).
	\]
	Also, \[
		v(G) + f(G) - e(G) = 2.
	\]
	It follows that	\[
		e(G) = 2n - 4.
	\]
\end{dem}

\begin{thm}
	$K_{3, 3}$ is not planar.
\end{thm}
\begin{dem}
	$K_{3, 3}$ is triangle-free.
	\[
		e(K_{3, 3}) = 9 > 2\cdot6 - 4 = 2v(K_{3, 3}) - 4
	\] 
\end{dem}

\begin{thm}
	$G$ is planar if, and only if, $G$ does not have a topological copy of $K_5$ or $K_{3, 3}$ if, and only if, $G$ does not have a $K_5$-minor or a $K_{3, 3}$-minor.
\end{thm}

\newpage
\section{More colors}

\begin{defn}[Chromatic Number of a Graph]
	The chromatic number of $G$, denoted by $\chi(G)$, is the smallest $r$ such that there is a coloring $c \colon V(G) \to [r]$ such that $c(u) \neq c(v)$ whenever $uv \in E(G)$.
\end{defn}


    \lecture{4}{January 07, 2021}{}
\begin{defn}
	Let $r(G, H)$ denote the minimum $n$ such that, for every coloration $c \colon E(K_n) \to \{R, B\}$, there must exist a red $G$ or a blue $H$.
\end{defn}

\begin{prop}
	\[
		\chi(G) \le \Delta(G) + 1.
	\]
\end{prop}

\begin{sk}
	Greedy algorithm.
\end{sk}

\begin{thm}[4-color Theorem, 1970's]
	If $G$ is planar, then $\chi(G) \le 4$.
\end{thm}

\begin{prop}
	If $G$ is planar, then $\chi(G) \le 6$.
\end{prop}

\begin{dem}
	Induction on $n$.

	Since $G$ is planar, $e(G) \le 3n - 6$, thus $\delta(G) \le 5$. Pick $v$ with degree at most $5$. 
	Define $G'$ as $G$ without $v$, then $G'$ has a proper coloring. Now, $v$ has at most five neighbors, thus we can pick one color for $v$ out of six such that no neighbor of $v$ has this color.
\end{dem}

\begin{figure}[ht]
    \centering
	\incfig[.8]{five-color-theorem}
    \caption{Five color theorem}
    \label{fig:second-case-five-color-theorem}
\end{figure}

\begin{thm}
	If $G$ is planar, then $\chi(G) \le 5$.
\end{thm}

\begin{dem}
	Induction on $n$.

	Since $G$ is planar, $e(G) \le 3n - 6$, thus $\delta(G) \le 5$. Pick $v$ with degree at most $5$. 
	Define $G'$ as $G$ without $v$, then $G'$ has a proper coloring. Now, $v$ has at most five neighbors. If there at most four colors are used in the neighbors of $v$, we can paint $v$ with a distinct color.

	Suppose all neighbors of $v$ have different colors. Let's call the neightbors $u_1, u_2, u_3, u_4, u_5$, in clockwise order, with colors $1, 2, 3, 4, 5$.

	Define $\left.G'\right._a^b$ as the subgraph of $G'$ that only contains vertices with colors $a$ and $b$. Let $H_a^b$ be the connected component of $\left.G'\right._a^b$ that contains $u_a$.
	\begin{enumerate}[label = \textbullet]
		\item \textbf{\boldmath If there exists $a, b$ such that $u_b \not\in H_a^b$,} then we flip the colors $a$ and $b$ inside $H_a^b$ and define $c(v) := a$.
		\item \textbf{\boldmath If, for all $a, b$, $u_b \in H_a^b$,} $H_{1, 3}$ and $H_{2, 4}$ are vertex disjoint, but have to go through each other; a contradiction. See \cref{fig:second-case-five-color-theorem}.
	\end{enumerate}
\end{dem}


\begin{thm}[Erd\H{o}s-Stone, 1946]
	\[
		\ex(n, H) = \left(1 - \frac{1}{\chi(H) - 1} + o(1)\right)\binom{n}{2}.
	\]
\end{thm}

\begin{sk}
	The example is the Turán's Graph $T_{\chi(H)-1}(n)$.
\end{sk}

    \newpage
\section{Ramsey's Theory again}
\lecture{5}{January 18, 2021}{YouTube, Lec. 5}

\begin{defn}
	Let $R_r^{(k)}(m)$ is the minimal  $n$ such that, for all colorings $c \colon \binom{[n]}{k} \to [r]$, there exists a monochromatic copy of $K_{m}^{(k)}$.
\end{defn}

We'll consider $r = 2$ and $k = 2$, if not otherwise stated.

\begin{rem}
	$K_m^{(k)}$ is the $k$-uniform complete hypergraph with $n$ vertices. $E\left(K_m^{(k)}\right) = \dbinom{V(K_n^{(k)})}{k}$. See \href{https://en.wikipedia.org/wiki/Hypergraph}{Wikipedia}.
\end{rem}

\begin{thm}[Ramsey, 1930]
	\[
		R_r^{(k)} (m) < \infty.
	\]
\end{thm}

\begin{sk}
	Induction on $k$.

	Pick $v_1 \in G$. Given $c \colon \binom{V(G)}{k} \to [r]$, define $c_1: \binom{v(G)\backslash \{v\}}{k-1}$. Induction hypothesis implies that there exists a monochromatic copy of $K_{m_1}^{(k-1)}$, for $n \ge R_r^{(k-1)}(m_1)$. 

	Repeat the process inside this copy of $K_{m-1}^{(k-1)}$.

	Similarly to the proof of \cref{thm:ramsey}, we'll have a sequence $v_1, v_2, \dots, v_\ell$ (that gets larger as $n$ gets larger), for which $c(\{v_{a_1}, v_{a_2}, \dots, v_{a_k}\}) = f(a_1)$, if $a_1 < a_2 < \dots < a_r$.

	Pick large $n$ such that $\ell \ge (r-1)m + 1$, for which there exists a subsequence $a_{b_1}, \dots, a_{b_r}$ such that $f(a_{b_i})$ is the same for all $i$.
\end{sk}

\begin{thm}[Erd\H{o}s--Hajnal]
	\[
		R^{(k)}(m) \le 2^{\dbinom{R^{{(k-1)}}(m)}{k-1}}
	\]
\end{thm}  

\begin{sk}[for $k = 3$]
	Suppose $e(G) \gtrsim 2^{\binom{R(m)}{2}}$

	Pick a edge $v_1v_2 \in E(G)$. Given $c \colon \binom{V(G)}{3} \to \{1, 2\}$, define $c' \colon \binom{V(G)\backslash\{v_1, v_2\}}{2} \to \{1, 2\}$ by  $c'(v) := c(v_1v_2v)$. The coloring $c'$ naturally partitions $V(G)\backslash\{v_1, v_2\}$ into two parts, one for each color --- denote the largest part by $A_3$, this has $\gtrsim n/2$ vertices. This implies that $c(v_1v_2v)$ is constant for all $v \in A_3$ --- denote this constant by $f(v_1v_2)$.

	Now, pick a vertex in $v_3 \in A_3$. Create similar colorings for the edges $v_1v_3$ and $v_2v_3$. There is a subset $A_4 \subset A_3$, with $\gtrsim n/8$ vertices, such that $c(v_1v_3v)$ and $c(v_2v_3v)$ are constant for all $v \in A_3$ --- denote those constants by $f(v_1v_3)$ and $f(v_2v_3)$.

	Repeat this process $R(m)$ times, which we can because $n \ge 2^{\binom{R(m)}{2}}$. Now, we have vertices $v_1, \dots, v_{\binom{R(m)}{2}}$, with a coloring $f$ of each $2$-edge, in which $f(v_{a_1}v_{a_2}) = c(v_{a_1}v_{a_2}v_{a_3})$, for all $a_1 < a_2 < a_3$. By definition, there is a monochromatic $K_m$ over the coloring $f$, which implies that there exists a monochromatic $K^{(3)}_m$ over the coloring $c$.
\end{sk}

\subsection{Happy Ending Problem}

\begin{prob}
	Given $5$ points on the plane, prove that there are $4$ of them that form a convex polygon.
\end{prob}

\begin{sol}
	If the convex hull has size $5$ or $4$, we're ok. If it has size $3$, then draw a line through the $2$ points inside the convex hull, it meets two of the three sides of the convex hull. The two points inside and the two points in the side not crossed form a convex polygon.
\end{sol}

\begin{defn}
	Let $f(k)$ be the minimal $n$ such that, for any set of $n$ points in $\RR^2$ in general position, there are $k$ points that form a convex polygon.
\end{defn}

\begin{thm}[Erd\H{o}s-Szekeres, 1935]
	\[
		f(k) \le R^{(4)}(k) \le 2^{2^{2^{ck}}}.
	\]
\end{thm}

\begin{dem}
	Suppose $n > R^{(4)}(k)$.

	Define $c \colon \binom{[n]}{4} \to {R, B}$ by $c(\{A, B, C, D\}) = R$ if, and only if, $\{A, B, C, D\}$ does form a convex polygon.

	By definition, there exists a monochoromatic $K^{(4)}_k$. For $k \ge 5$, it cannot be blue. Therefore, it's red, which would not be possible if those $k$ vertices didn't form a convex polygon.
\end{dem}

\subsection{Monochromatic Arithmetic Progression}

\begin{defn}
	Let $W(r, k)$ be the minimal $n$ such that for all $c \colon [n] \to [r]$, there exists a monochromatic arithmetic progression of size $k$.
\end{defn}

\begin{thm}[Van der Waerden, 1927]\label{thm:vanderwaerden-1927}
	Let $c \colon \NN \to [r]$. There is a monochromatic arithmetic progression of size $k$, for all positive integers $k$.

	Equivalently, \[
		W(r, k) < \infty.
	\]
\end{thm}

\begin{defn}
	Denote $\left\{a, a + d, a + 2d, \dots, a + (k-1)d\right\}$ by  $PA_k(a, d)$.

	The arithmetic progressions  $PA_k(a_1, d_1), PA_k(a_2, d_2) \dots, PA_k(a_s, d_s)$ are color-focused if:
	\begin{enumerate}
		\item They are monochromatic with diferent colors.
		\item They have the same ``focus'' $f$, i.e., \[
			a_1 + kd_1 = \cdots = a_s + kd_s = f
		\]
	\end{enumerate}
\end{defn}

\begin{dem}[of \nameref{thm:vanderwaerden-1927}]
	We will use induction on $k$. Note that $W(r, 1) = 1$.

	We shall find $r$ color-focused $(k-1)$-arithmetic progressions.
	\begin{lem}
		There exists $n = n(s, r)$ such that, for every coloring $c\colon [n] \to [r]$, there exists a monochromatic $k$-arithmetic progression or $s$ color-focused $(k-1)$-arithmetic progressions.
	\end{lem}
	\begin{dem}
		Induction on $s$. $n(1, r) = W(r, k-1) < \infty$.

		Let $N = 2n(s-1, r)$. Consider $W(r^N, k-1) < \infty$ blocks of size $N$. There is an arithmetic progression of equally-colored blocks of size $k-1$, let $D$ be the distance of consecutive blocks in the arithmetic progression of blocks. Since the first half of the block has $n(s-1, r)$ elements, there exists a monochromatic $k$-arithmetic progression (which means we're done), or $s-1$ color-focused $(k-1)$-arithmetic progressions -- their focus $f$ surely lies inside the block of size $N$.

		Let the $s-1$ color-focused $(k-1)$-arithmetic progressions in the first block be $PA_{k-1}(a_1, d_1), \dots, PA_{k-1}(a_{s-1}, d_{s-1})$, with focus $f_1$. The proposed $s$ color-focused $(k-1)$-arithmetic progressions are $PA_{k-1}(a_1, d_1 + d), \dots, PA_{k-1}(a_{s-1}, d_{s-1} + d), PA_{k-1}(f_1, d)$.

		Therefore,  \[
			n(s, r) \le 2 \cdot W(r^{2n(s-1, r)}, k-1) \cdot 2n(s-1, r).
		\]
	\end{dem}

	Therefore, for suitable large $n$, there must exist a large $k$-arithmetic progression.
\end{dem}

    \newpage\section{Problemas extremais para sistemas de conjuntos}
\lecture{6}{January 19, 2021}{YouTube, Lec. 7}

\begin{defn}
	$\mathcal{A} \subset \mathcal{P}([n])$ is an \emph{anti-chain} se  $A \not\subset B$, para todo $A, B \in \mathcal{A}$, $A \neq B$.
\end{defn}

\begin{thm}[Sperner, 1910's] \label{thm:sperner}
	$\mathcal A \subset \mathcal P([n])$ é uma anti-cadeia $\implies |\mathcal A| \le \binom{n}{n/2}$.
\end{thm}

The example is $\binom{[n]}{n/2}$.

\begin{lem}[LYMB, 1960's]\label{lem:lymb}
	$\mathcal A \subset \mathcal P([n])$ é uma anti-cadeia $\sum_{A \in \mathcal A} \frac{1}{\binom{n}{|A|}} \leq 1$.
\end{lem}

\begin{dem}[of \nameref{thm:sperner}]
	We know that $\binom{n}{k} \le \binom{n}{n/2}$. Thus, by \nameref{lem:lymb}, \[
		1 \ge \sum_{A\in \mathcal{A}} \frac{1}{\binom{n}{|A|}} \ge \sum_{A\in \mathcal{A}} \frac{1}{\binom{n}{n/2}} = \frac{|\mathcal{A}|}{\binom{n}{n/2}}
	\]
\end{dem}

\begin{dem}[of \nameref{lem:lymb}]
	Let's count the pairs $(\pi, A)$ such that $\pi$ is a permutation of $[n]$, $A \in \mathcal{A}$, and  $\{\pi(1), \pi(2), \dots, \pi(|A|)\} = A$.

	For each $A \in \mathcal{A}$, the number of $\pi$ such that $\{\pi(1), \pi(2), \dots, \pi(|A|)\}$ is equal to $|A|!(n - |A|)!$.

	For each $\pi$, the number of $A \in \mathcal{A}$ such that $\{\pi(1), \dots, \pi(|A|)\}$ is at most $1$, since $\mathcal{A}$ is an anti-chain.

	Therefore, \[
		\sum_{A \in \mathcal{A}} |A|!(n-|A|)! \le n!
		\implies
		\sum_{A \in \mathcal{A}} \frac{1}{\binom{n}{|A|}} \le 1.
	\]
\end{dem}

\begin{defn}
	$\mathcal{A}$ is \emph{intersecting} if $A \cap B \neq \varnothing$, for all  $A, B \in \mathcal{A}$.
\end{defn}

\begin{prop}
	$\mathcal{A} \subset \mathcal{P}([n])$ is intersecting $\implies |A| \le 2^{n-1}$.
\end{prop}

\begin{sk}
	At most one of $(S, \overline{S})$ can be on $\mathcal{A}$.
\end{sk}

\begin{thm}[Erd\H{o}s-Ko=Rado, 1961]
	$\mathcal{A} \subset \binom{[n]}{k}$ is intersecting $\implies |A| \le \binom{n-1}{k-1}$, for $k < \frac{n+1}{2}$.
\end{thm}

\begin{dem}
	Let's count the number of pairs $(\pi, A)$ such that $\pi$ is a circular permutation and $A \in \mathcal{A}$ is an interval in $\pi$.

	For each $A \in \mathcal{A}$, the number of permutations such that $A$ is an interval in $\pi$ is $k!(n-k)!$

	For each circular permutation $\pi$, the number of $A \in \mathcal{A}$ such that $A$ is an interval in $\pi$ is at most $k$.

	Therefore, \[
		|A|k!(n-k)! \le (n-1)!k
		\implies
		|A| \le \binom{n-1}{k-1}.
	\]
\end{dem}

    \newpage
\section{Supersaturation and Stability}
\lecture{7}{January 19, 2021}{Youtube, Lec. 10}

\begin{defn}
	$G$ is \emph{$t$-close to bipartite} if there exists $T \subset E(G)$, $|T| \le t$ such that $G - T$ is bipartite.
	
	Otherwise, $G$ is \emph{$t$-far from bipartite.}
\end{defn}

\begin{thm}[F\"{u}redi]\label{thm:furedi}
	If $G$ is $t$-far from bipartite, then \[\#\ K_3 \text{ in } G \ge \frac{n}{6}\left(e(G) + t - \frac{n^2}{4}\right).\]
\end{thm}

\begin{dem}
	Let $N(v)$ be the neighborhood of $v$. Then, \[
		\# K_3 \text{ in } G = \frac{1}{3} \sum_{v \in G}e(N(v)).
	\]

	Also, since $G$ is $t$-far from bipartite,  \[
		e(N(v)) + e(\overline{N(v)}) > t
	\]

	Lastly, \begin{align*}
		\sum_{u \in \overline{N(v)}} d(u) &= e(\overline{N(v)}, N(v)) + 2e(\overline{N(v)})\\
											 &= e(G) + e(\overline{N(v)}) - e(N)\\
											 &> e(G) + t - 2e(N(v)).
	\end{align*}

	Therefore, \begin{align*}	
		\# K_3 \text{ in } G &= \frac{1}{3} \sum_{v \in G}e(N(v))\\
							 &> \frac{1}{6} \sum_{v \in G} \left( e(G) + t - \sum_{u \in \overline{N(v)}} d(u)\right)\\
							 &> \frac{1}{6} \sum_{v \in G} \left( e(G) + t\right)  - \frac{1}{6} \left(\sum_{v\in G}\sum_{u \in \overline{N(v)}} d(u)\right)\\
							 &> \frac{n}{6} \left( e(G) + t\right)  - \frac{1}{6} \sum_{\substack{v\in G\\ u\in G\\ u \not\sim v}} d(u)\\
							 &> \frac{n}{6} \left( e(G) + t\right)  - \frac{1}{6} \sum_{\substack{u \in G}} d(u)(n - d(u))\\
							 &> \frac{n}{6} \left( e(G) + t\right)  - \frac{1}{6} \frac{n^3}{4}\\
							 &> \frac{n}{6} \left( e(G) + t - \frac{n^2}{4}\right)\\
	\end{align*}
\end{dem}

\begin{cor}
	$e(G) \ge \frac{n^2}{4} + t \implies \# K_3 \text{ in } G \ge \frac{tn}{3}$.
\end{cor}

\begin{cor}
	$e(G) > \frac{n^2}{4} - t$, $K_3 \not\subset G \implies G$ is $t$-close to bipartite.
\end{cor}

\begin{thm}[Generalization of F\"{u}redi]
	If $G$ is $t$-far from $r$-partite, then
	\[
		\# K_{r+1}\text{ in } G \ge c(r) n^{r-2} \left( e(G) + t - \left(1 - \frac{1}{r}\right)\binom{n}{2} \right).
	\]
\end{thm}

\begin{dem}
	Left as an exercise. Same idea; use induction; use H\"older.
\end{dem}

\begin{thm}[Stability Theorem of Erd\H{o}s and ---, 1970's]
	Let $H$ be a graph. For all $\varepsilon > 0$, there is $\delta > 0$ such that the following property holds. 

	If $H \not\subset G$ and \[ e(G) \ge \left(1 - \frac{1}{\chi(H) - 1} - \delta \right) \binom{n}{2},\] then $G$ is $\varepsilon n^2$-close to $(\chi(H) - 1)$-partite.
\end{thm}

\begin{sk}
	For simplicity, let $\chi(H) = 3$. Thus, we shall prove $e(G) \ge \frac{n^2}{4} - \delta n^2 \implies G$ is $\varepsilon n^2$-close to bipartite.

	Let's have $H \subset K_3(s)$, for some $s$. $G$ is $\varepsilon n^2$-far from bipartite, and $e(G) \ge \frac{n^2}{4} - \delta n^2$. We shall prove that $K_3(s) \subset G$.

	By \nameref{thm:furedi}, for $t = \varepsilon n^2$, 
	\begin{align*}
		K_3(G) &\ge \frac{n}{6} \left(e(G) + \varepsilon n^2 - \frac{n^2}{4} \right)\\
			   &\ge \frac{\varepsilon n^3}{12}.
	\end{align*}

	By \cref{thm:k-hypergraph:erdos}, we're done!

	%TODO
\end{sk}

\begin{thm}[Erd\H{o}s]\label{thm:k-hypergraph:erdos}
	Let $H$ be a $k$-uniform hypergraph. if $e(H) \ge \alpha \binom{n}{k}$, then there exists a copy of $K^{(k)}_{k}(t)$, the complete $k$-partite hypergraph, inside $H$.
\end{thm}

\begin{sk}

\end{sk}

%\begin{thm}[Erd\H{o}s]\label{thm:3-hypergraph:erdos}
%	Let $H$ be a $3$-uniform hypergraph. if $e(H) \ge \varepsilon n^3$, então $K^{(3)}_{t, t, t} \subset H$.
%\end{thm}
%
%\begin{sk}
%	Remove hyperedges that contain pairs with low degree in $H$.
%
%	Count some generalization of cherries.
%	%TODO
%\end{sk}

    \newpage
\section{Random Graphs and Thresholds}
\lecture{8}{January 27, 2021}{YouTube, Lec. 6}

In this section, we'll recall some things we've seen before. Namely, the \nameref{defn:randomgraph} and \nameref{lem:markovinq}. Another inequality that will be useful is \nameref{lem:chebychevinq}.

\begin{lem}[Chebychev's inequality]\label{lem:chebychevinq}
	\[
		\Prob(|X - \Exp[X]| \ge t ) \le \frac{\Var(X)}{t^2}.
	\]
\end{lem}

\begin{defn}[Variance]\label{defn:variance}
	\begin{align*}
		\Var(X) &= \Exp\left[\left(X - \Exp[X]\right)^2\right] \\
				&= \Exp\left[X^2\right] - \Exp[X]^2
	\end{align*}
\end{defn}

\begin{sk}[of \nameref{lem:chebychevinq}]
	Apply \nameref{lem:markovinq} with $(X-\Exp(X))^2$ and $t^2$.
\end{sk}

\subsection{Triangle-free}

\begin{prop}
	If $p \ll \frac{1}{n}$, i.e., $pn \to 0$, then $\Prob(\text{copy of } K_3 \subset G(n, p)) \to 0$, in other words, $K_3 \not\subset G(n, p)$ with high probability.
\end{prop}

\begin{dem}
	Define $X$ as the number of copies of $K_3$ in $G(n, p)$. \[ \Exp[X] = \binom{n}{3} p^3 \le n^3p^3 \to 0.\]
	Therefore, \[ \Prob(X \ge 1) \to 0 \implies \Prob(X = 0) \to 1. \]	
\end{dem}

\begin{prop}
	If $p \gg \frac{1}{n}$, i.e., $pn \to \infty$, then $\Prob(\text{copy of } K_3 \subset G(n, p)) \to 1$.
\end{prop}

\begin{dem}
	Again, define $X$ as the number of copies of $K_3$ in $G(n, p)$. First, $\Exp[X]^2 = \binom{n}{3}^2p^6 \to \infty$. Second,
	\begin{align*} \Exp\left[X^2\right] &= \sum_{\substack{S, T\text{ copies}\\\text{of }K_3}} \Prob(S \subset G(n, p) \text{ and } T \subset G(n, p)) \\
		&\le \binom{n}{3}^2p^6 + n^4p^5 + n^3p^3.
	\end{align*}

	Therefore, $\Var(X) \le n^4p^5 + n^3p^3 \ll \Exp[X]^2$.
	
	Applying \nameref{lem:chebychevinq} with $t = \frac{\Exp[X]}{2}$, we have
	\[
		\Prob\left(X \le \frac{\Exp[X]}{2}\right) \le \frac{4\Var(X)}{\Exp[X]^2} \to 0.
	\]
\end{dem}

This means that $\frac{1}{k}$ is a threshold. If $p$ is much smaller than $\frac{1}{n}$, then there is no triangle with high probability. If $p$ is much bigger than $\frac{1}{n}$, then there is a triangle with high probability. 

Can we do the same for $K_r$? Let's try.

\begin{prop}
	If $p \ll n^{-\frac{2}{r-1}}$, then $\Prob(\text{copy of } K_r \subset G(n, p)) \to 0$.
\end{prop}

\begin{dem}
	Define $X$ as the number of copies of $K_r$ in $G(n, p)$. \[ \Exp[X] = \binom{n}{r} p^{\binom{r}{2}} \to 0.\]
	Therefore, \[ \Prob(X \ge 1) \to 0.\]	
\end{dem}


\begin{prop}
	If $p \gg n^{-\frac{2}{r-1}}$, then $\Prob(\text{copy of } K_r \subset G(n, p)) \to 1$.		
\end{prop}

\begin{dem}
	Again, define $X$ as the number of copies of $K_r$ in $G(n, p)$. First, $\Exp[X]^2 = \left(\binom{n}{r}p^{\binom{r}{2}}\right)^2 \to \infty$. Second,
	\begin{align*} \Exp\left[X^2\right] &= \sum_{\substack{S, T\text{ copies}\\\text{of }K_r}} \Prob(S \subset G(n, p) \text{ and } T \subset G(n, p)) \\
		&\le \left(\binom{n}{r}p^{\binom{r}{2}}\right)^2 + \sum_{k=2}^r n^{2r-k} p^{2\binom{r}{2} - \binom{k}{2}}.
	\end{align*}

	Note that, for $2 \le k \le r$, $n^{2r-k}p^{2\binom{r}{2} - \binom{k}{2}} \ll n^{2r}p^{2\binom{r}{2}} \approx \Exp[X]^2$.

	Therefore, $\Var(X) \le \sum_{k=2}^r n^{2r-k} p^{2\binom{r}{2} - \binom{k}{2}}\ll \Exp[X]^2$.
	
	Applying \nameref{lem:chebychevinq} with $t = \frac{\Exp[X]}{2}$, we have
	\[
		\Prob\left(X \le \frac{\Exp[X]}{2}\right) \le \frac{4\Var(X)}{\Exp[X]^2} \to 0.
	\]
\end{dem}

    \subsection{Mathings}
\lecture{9}{January 28, 2021}{YouTuhe, Lec. 8}

\begin{defn}[Matching]
	A \emph{matching} is a graph in which each vertex has degree at most $1$.

	A \emph{perfect matching} is a graph in which each vertex has degree $1$.
\end{defn}

\begin{thm}[Dilworth]
	Let $P$ be a partially ordered set with $n$ vertices. Then, there exists a chain of size $k$ (i.e., a sequence $v_1, v_2, \dots, v_k$ such that $v_i < v_j$ whenever $i < j$) or an anti-chain of size $n/k$ (i.e., a set of vertices such that for any pair $u, v$, $u \not< v$).
\end{thm}

\begin{thm}
	Let $G$ be a bipartite graph with $n$ vertices on each part. Let's call the parts $A$ and $B$.

	There exists a perfect matching inside $G$ if, and only if, for all subsets $S \subset A$, \[
		N(S) := \left|\cup_{u \in S}N(u)\right|	\ge |S|.
	\]
\end{thm}

\begin{sk}
	It is clear that it is a necessary condition. We shall prove that it is a sufficient condition. We'll use induction on $n$.

	Suppose that there is a set $A_1 \subset A$, $A_1 \neq A, \varnothing$ such that $|N(A_1)| = |A_1|$. Consider the graphs $G_1$ and $G_2$ by restraining the vertices to $A_1 \cup N(A_1)$ and $\overline{A_1 \cup N(A_1)}$, respectively. Show that $G_1$ and $G_2$ satisfy the hypothesis. Imply that there is a matching inside $G$.

	Suppose that $|N(S)| > |S|$ for all $S \subset A$, $S \neq A, \varnothing$. Pick pick any edge $uv$ and fix it. Consider  $G' = G - u - v$. Show that $G'$ satisfy the hypothesis. Imply that there is a matching inside $G$.
\end{sk}

\begin{prop} \label{prop:isolatedvertex}
	If $p < (1-\varepsilon)\frac{\log n}{n}$, then there exists an isolated vertex in $G(n, p)$ with high probability.
\end{prop}

\begin{dem} Let $X$ denote the number of isolated vertices in $G(n, p)$.
	\begin{align*}
		\Exp[X] &= n(1-p)^{n-1} \\
				&\gtrsim ne^{-(1-\varepsilon)\log{n}}\\
				&\gtrsim n^\varepsilon.
	\end{align*}

	On the other hand, 
	\begin{align*}
		\Exp[X^2] &= \sum_{u, v} \Prob(u, v \text{ are isolated})\\
				  &= n(n-1)(1-p)^{2n-3} + n(1-p)^{n-1}\\
				  &= \Exp[X]^2 \frac{n-1}{n}(1-p)^{-1} + \Exp[X].
	\end{align*}

	Therefore, 
	\begin{align*}
		\Var(X) &= \Exp[X^2] - \Exp[X]^2\\
				&\le 2p\Exp[X]^2 + \Exp[X]\\
				&\ll \Exp[X]^2.
	\end{align*}

	Thus, by \nameref{lem:chebychevinq}, we're done.
\end{dem}

\begin{thm}
	Suppose $n$ is even.

	If $p < (1-\varepsilon)\frac{\log{n}}{n}$, then there is no perfect matching in $G(n, p)$ with high probability.

	If $p > (2+\varepsilon)\frac{\log{n}}{n}$, then there is a perfect matching in $G(n, p)$ with high probability.
\end{thm}


    \newpage
\section{Conectivity and Hamiltonian cycles}
\lecture{10}{January 28, 2021}{YouTube, Lec. 9}

\begin{thm}
	If $p < (1-\varepsilon)\frac{\log{n}}{n}$, then $G(n, p)$ is not connected with high probability.

	If $p > (1+\varepsilon)\frac{\log{n}}{n}$, then $G(n, p)$ is connected with high probability.
\end{thm}
\begin{dem}[of the first part]
	Directly from \cref{prop:isolatedvertex}.
\end{dem}
\begin{dem}[of the second part]
	A graph $G$ is disconnected if, and only if, there exits a complete bipartite graph which is a subgraph of $\overline G$. 

	For $k \in \{1, \dots, n/2\}$, let $X_k$ be the number of copies of $K_{k, n-k}$ in $\overline{G(n,p)}$.
	\begin{align*}
		\Exp[X_k] &= \binom{n}{k}(1-p)^{k(n-k)} \\
				  &\le \left(\frac{en}{k}e^{-p(n-k)}\right)^k \\
				  &\le \left(\frac{en}{k}n^{-\left(1+\epsilon\right)\left(1-\frac{k}{n}\right)}\right)^k\\
				  &\le n^{-\epsilon k /2} \to 0
	\end{align*}

	Since $X_k = 0$, for $k \in \{1, \dots, n/2\}$ with high probability, then $G(n,p)$ is connected with high probability.
\end{dem}

\begin{defn}[Sharp threshold]
	An event $\mathcal A = \mathcal A(n)$ has a \emph{sharp threshold} if there exists $p_c$ such that:
	\begin{enumerate}[label = \textbullet] 
		\item $p \ge (1+\epsilon)p_c \implies \PP(\mathcal A) \to 1$, as  $n \to \infty$;
		\item $p \ge (1-\epsilon)p_c \implies \PP(\mathcal A) \to 0$, as  $n \to \infty$.
	\end{enumerate}
\end{defn}

\begin{defn}[Coarse threshold]
	An event $\mathcal A = \mathcal A(n)$ has a \emph{coarse threshold} if there exists $p_c$ such that:
	\begin{enumerate}[label = \textbullet] 
		\item $p \gg p_c \implies \PP(\mathcal A) \to 1$, as  $n \to \infty$;
		\item $p \ll p_c \implies \PP(\mathcal A) \to 0$, as  $n \to \infty$.
	\end{enumerate}
\end{defn}

\begin{thm}[Bollobás--Thomason, 1980s]
	Every increasing property (in the sense of adding edges) has a coarse threshold.
\end{thm}

%31min

	% end lectures
\end{document}
