\lecture{2}{January 06, 2021}{}

Let's now try to find a lower bound.
It is very difficult to show a good construction.
Luckly, we are not going to do that.

\begin{thm}[Erd\H{o}s, 1947]
	\[
		\left.\sqrt{2}\right.^k \le R(k)
	\]
\end{thm}

\begin{dem}
	Let $n \le \left.\sqrt{2}\right.^k$.
	Let's choose colors randomly. Let \[
		\Prob(c(e) = R) = \frac{1}{2},
	\]
	for every edge $e$ in $K_n$, independently.

	We want to show that \[
		\Prob(\text{there is not a monochromatic copy of $K_k$}) > 0.
	\]

	Let $X$ be the number of monochromactic copies of $K_k$ in $c$. Then, 
	\begin{align*}
		\Exp[X] &= \Exp\left[\sum_{\substack{S \subset K_n\\S\ \text{is a copy of}\ K_k}} \Ind[S\ \text{is monochromatic}]\right]\\
			    &= \sum_{\substack{S \subset K_n\\S\ \text{is a copy of}\ K_k}} \Exp\left[\Ind[S\ \text{is monochromatic}]\right]\\
				&= \sum_{\substack{S \subset K_n\\S\ \text{is a copy of}\ K_k}} \Prob(S\ \text{is monochromatic}))\\
				&= \sum_{\substack{S \subset K_n\\S\ \text{is a copy of}\ K_k}} \left(\frac{1}{2}\right)^{\binom{k}{2} - 1}\\
				&= \binom{k}{n} \left(\frac{1}{2}\right)^{\binom{k}{2} - 1}\\
				&\le 2 \left(\frac{en}{k}\right)^k \left(\frac{1}{2}\right)^{\frac{k(k-1)}{2}}\\
				&\le 2 \left(\frac{e\sqrt{2}}{k}\right)^k \\
				&< 1\text{, for $k \ge 5$.}
	\end{align*}

	Therefore, since $\Exp[X] < 1$, we have $\PP(X = 0) > 0$.
\end{dem}

The bounds have not improved much since then 

\begin{thm}[Conlon, 2009]
	\[
		R(k) \le \frac{4^k}{k^{\sqrt{\log k }}}
	\]
\end{thm}

\newpage
\section{Extremal Graph Theory}
\subsection{Complete Graphs}

\begin{defn}
	Let $\ex(n, H)$ be the maximum number of edges a graph  $G \subset K_n$ can have such that there are no copies of $H$ in $G$.
\end{defn}

\begin{thm}[Mantel, 1907]
	\[
		\ex(n, K_3) = \floor{\frac{n^2}{4}}.
	\]
\end{thm}

\begin{dem}
	The example is the bipartite graph with $\floor{\frac{n}{2}}$ and $\ceil{\frac{n}{2}}$ vertices.

	Let's prove by indution on $n$.

	Now, suppose $G$ does not have a triangle.
	Pick an edge $uv$. Let $G'$ be the graph $G$ deleting $u$ and $v$. The subgraph $G'$ also does not contain triangles, so $e(G') \ge \floor{\frac{n^2}{4}}$.

	Notice that cannot exist $w \in G'$ such that $uw$ and $vw$ are edges of $G$, because $G$ does not have triangles. Therefore, there can be at most $n-2$ edges from $u$ or $v$ to vertices on $G'$. Including the edge $uv$, we conclude that
	\begin{align*}
		e(G) &\le e(G') + n - 1\\
			 &\le \floor{\frac{(n-2)^2}{4}} + n - 1\\
			 &\le \floor{\frac{n^2}{4}}.
	\end{align*}
\end{dem}

\begin{figure}[ht]
    \centering
	\incfig[.6]{edge-uv-on-a-triangle-less-graph}
    \caption{Edge $uv$ on a triangle-free graph.}
    \label{fig:edge-uv-on-a-triangle-less-graph}
\end{figure}

\begin{defn}[Turán's Graph]
	The graph $T_r(n)$ consists of $r$ sets with roughly $n/r$ elements each (some rounded up, some rounded down).; we create an edge $uv$ if, and only if, $u$ and $v$ are on different sets.

	We'll denote by $t_r(n)$ the number of edges in $T_r(n)$.
\end{defn}

\begin{thm}[Turán, 1941]
	\[
		\ex(n, K_{r+1}) = t_r(n) \approx \left(1 - \frac{1}{r}\right)\binom{n}{2}.
	\]
\end{thm}

\begin{dem}
	We'll use induction on $n$. For $n \le r$, we're good.

	Pick a maximal graph $G$ that doesn't have a copy of $K_{r+1}$. Pick a copy of $K_r$, let's call it $H$. Define $G' = G - H$. Of course, $G'$ has no copies of $K_r$; thus $e(G') \le t_r(n-r)$, by induction.

	Futhermore, if $v \in G'$, there can be at most $r - 1$ edges connecting $v$ to some vertex in $H$.

	Wrapping everything up, we have
	\begin{align*}
		e(G) &\le e(G') + (n-r)(r-1) + \binom{r}{2}\\
			 &\le t_r(n-r) + (n-r)(r-1) + \binom{r}{2}\\
			 &\le t_r(n).
	\end{align*}
\end{dem}

\begin{figure}[h]
    \centering
	\incfig[.5]{grafo-de-turan}
    \caption{Turán's Graph}
    \label{fig:grafo-de-turan}
\end{figure}

\subsection{Bipartite Graphs}

\begin{thm}[Erd\H{o}s, 1935]
	\[
		\ex(n, C_4) \le \frac{n^{3/2}}{2}.
	\]
\end{thm}

\begin{dem}
	Let's count cherries! A \emph{cherry} is a pair $(v, \{u, w\})$, in which $vu$ and $vw$ are edges of the graph. 

	Since there is no $C_4$, there is at most one cherry for each pair $\{u, w\}$. This implies that:
	 \begin{align*}
		 \binom{n}{2} \ge \#(\text{cherries}) &= \sum_{v \in G} \binom{d(v)}{2}\\
			&\ge n\binom{\frac{2e(G)}{n}}{2}.
	\end{align*}

	Solving this quadractic inequation on $e(G)$ yields to  \[
		e(g) \ge \frac{n^{3/2}}{2}.
	\]
\end{dem}

\begin{ques}
	For which graphs we have \[
		\ex(n, H) = \Theta(n^2)?
	\]
\end{ques}

\begin{prop}
	For every non-bipartite graph $H$, we have \[
		\ex(n, H) \ge \frac{n^2}{4}.
	\]
\end{prop}

\begin{dem}
	Take $G$ as the complete bipartite graph with $n$ vertices. It has roughly $\frac{n^2}{4}$ vertices and it cannot contain a non-bipartite graph.
\end{dem}

\begin{thm}[Kővári–Sós–Turán, 1954]\label{thm:kovari-sos-turan-1954}
	Let $H$ be a bipartite graph. Then, \[
		\ex(n, H) = o(n^2).
	\]
\end{thm}

\begin{dem}
	Since $H$ is bipartite, there is some $K_{s,t}$ such that $H \subset K_{s, t}$. Then, \[
		\ex(n, H) \le \ex(n, K_{s, t}).
	\]

	Let's bound $\ex(n, K_{s, t})$.

	We'll count $s$-cherries: $(v, S)$, in which $S$ has size $s$ and $vx \in E(G)$ for all $x \in S$.


	There are at most $t-1$ $s$-cherries for each subset $S$ with size $s$. This implies that:
	 \begin{align*}
		 (t-1)\binom{n}{s} \ge \#(\text{cherries}) &= \sum_{v \in G} \binom{d(v)}{s}\\
			&\ge n\binom{\frac{2e(G)}{n}}{s} \ge \frac{e(G)^s}{s^s\cdot n^{s-1}}.
	\end{align*}

	This implies that, for some constant $C$, \[
		e(G) \le C \cdot n^{2 - \frac{1}{s}}
	\]

\end{dem}

\begin{ques}
	For which $H$ it holds that \[
		\ex(n, H) = O(n)?
	\]
\end{ques}
