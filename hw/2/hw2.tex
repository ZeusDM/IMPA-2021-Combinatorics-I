\documentclass[10pt, a4paper]{article}
\usepackage[utf8]{inputenc}
\usepackage[english]{babel}
\usepackage{fullpage}
\usepackage{parskip}
\usepackage[prob-boxed, dem-boxed]{zeus}

\DeclareMathOperator\ex{ex}

\title{\textbf{\textsf{Combinatorics I, List 2}}}
\author{\textsc{Guilherme Zeus Dantas e Moura}\\[2pt]\href{mailto:zeusdanmou@gmail.com}{\texttt{zeusdanmou@gmail.com}}}
\date{IMPA, Summer 2021}

\newcommand{\incfig}[2][1]{%
    \def\svgwidth{#1\columnwidth}
    \import{./figures/}{#2.pdf_tex}
}

\begin{document}
\maketitle

% PROBLEM 1
\begin{prob}
	Prove the following supersaturaion theorem for cliques:

	A graph with $o(n^r)$ copies of $K_r$ has at most $\ex(n, K_r) + o(n^2)$ edges.
\end{prob}

\begin{sk}
	Seja $B = \{T \subset V(G) : |T| = t, K_r \subset G[T]\}$.
	\[ |B| \le \epsilon n^k \cdot n^{t-k}.\]

	Vamos contar os pares de $(T, e)$ onde $T \subset V(G), |T| = t, e \in e(G(T))$. \[
		e(G)\binom{n-2}{t-2} = \#(T, e) \le (\epsilon n^t) \binom{t}{2} + \binom{n}{t} t_{r-1}(t).
	\]

	%Turan na segunda cota.
\end{sk}

% PROBLEM 2
\newpage
\begin{prob}
	Show that, for every graph $H$ and $\varepsilon > 0$, there exists $\delta > 0$ such that the following golds for all sufficiently large $n \in \NN$.

	If $G$ is a graph on $n$ vertices with \[ e(G) > (1- \delta) \binom{n}{2}, \] then in every $r$-colouring of $E(G)$ there are at least $\varepsilon n^{v(H)}$ monochromatic copies of $H$.	
\end{prob}
\begin{sk}
	Pegar $k$ grande suitable.

	Usar o item anterior pra achar muitos $K_k$.

	Fazer contagem dupla de $(C_{K_k}, C_H)$,  $C_H \subset C_{K_k}$.
\end{sk}

% PROBLEM 3
\begin{prob}
	Shay that a $k$-uniform hypergraph $G$ is said to be \emph{$2$-colourable} if there exists a partition $V(G) = A \cup B$ with no edges entirely contained in either $A$ or $B$. Let $b(k)$ denote the minimum number of edges $k$-uniform hypergraph that is not $2$-colourable.
	\begin{enumerate}[label = (\alph*)]
		\item By considering a random coloring, show that $b(k) \ge 2^{k-1}$.
		\item By considering a random hypergraph, probe an upper bound for $b(k)$.
	\end{enumerate}
\end{prob}
\begin{sk}[for (a)]
	It suffices to show that $k$-uniform hypergraph with $2^{k-1} - 1$ edges is $2$-colourable.

	Pick a random bipartition, $p = 1/2$. \[
		\Prob[e \text{ is monochromatic}] = \left(\frac{1}{2}\right)^{k-1}.
	\]

	Then, \[
		\Prob[\exists e \text{ monochromatic}] = \frac{v(\mathcal H)}{2^{k-1}} < 1.
	\]
\end{sk}
\begin{sk}[for an upper bound]
	The $k$-uniform complete hypergraph with $2k-1$ edges works.
\end{sk}
\begin{sk}[for (b)]
	Escolha um hypergrafo $k$-uniforme com $m$ arestas aleatório. Para facilitar, escolha as $m$ arestas independentemente, e se calhar de ser a mesma, o seu grafo vai ter menos arestas.

	Let $(A, B)$ be a bipartition. Say $|A| = a$. 
	 \begin{equation*}
		 \Prob(e \text{ is monochromatic}) = \frac{\binom{r}{k}}{\binom{n}{k}} + \frac{\binom{n-r}{k}}{\binom{n}{k}} \ge \frac{2\binom{n/2}{k}}{\binom{n}{k}}.
	\end{equation*}

	Logo, \[
		\Prob(\text{todas arestas não são monocromáticas}) = \left( 1 - \frac{2\binom{n/2}{k}}{\binom{n}{k}}\right)^m.
	\]

	Note que, vale
	\begin{align*}
		2^n \left(1 - \frac{\binom{n/2}{k}}{\binom{n}{k}}\right)^m > 1
	\end{align*}
	se $m > ?$.
\end{sk}

% PROBLEM 4
\begin{prob}
	Show that any finite set $A$ of integers contains a sum-free subset of size at least $|A|/3$.
\end{prob}
\begin{sol}
	Mergulha em $\ZZ/p\ZZ$, para $p$ primo suficientemente grande.

	Escolhe $x \in \ZZ/p\ZZ$ aleatório. Calcula o esperado de elementos de $xA$ que são $\{p/3, \dots, 2p/3\}$.
\end{sol}

% PROBLEM 5
\newpage
\begin{prob}
	
\end{prob}


% PROBLEM 6
\newpage
\begin{prob}
	Let $(e_1, \dots, e_m)$ be an arbitrary ordering of the edges of a graph $G$ on $n$ vertices. Show that there exists an incresing walk (in this ordering) of length at least $d = 2m/n$.
\end{prob}

\begin{sk}
	Coloca uma pessoa em cada vértice.

	Visite as arestas da menor para a maior, cada pessoa em um endpoint dessa aresta anda pro outro endpoint.

	No total, tiveram $2m$ passos.

	Por P.C.P., alguma pessoa andou pelo menos $2m/n$ passos.
\end{sk}

% PROBLEM 7
\newpage
\begin{prob}
	
\end{prob}
\begin{sk}
	Troca cada intersecção por um vértice. Agora o grafo é planar.
\end{sk}
\end{document}
