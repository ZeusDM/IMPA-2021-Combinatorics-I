\documentclass[10pt, a4paper]{article}
\usepackage[utf8]{inputenc}
\usepackage[english]{babel}
\usepackage{fullpage}
\usepackage{parskip}

\providecommand\styleidea{thmgraybox}
\usepackage[prob-boxed, dem-boxed]{zeus}

\DeclareMathOperator\ex{ex}

\title{\textbf{\textsf{Combinatorics I, Exam 1}}}
\author{\textsc{Guilherme Zeus Dantas e Moura}\\[2pt]\href{mailto:zeusdanmou@gmail.com}{\texttt{zeusdanmou@gmail.com}}}
\date{\vspace{-1ex}IMPA, Summer 2021}

\newcommand{\incfig}[2][1]{%
    \def\svgwidth{#1\columnwidth}
    \import{./figures/}{#2.pdf_tex}
}

\usepackage{mathtools}
\newcommand{\signexpl}[3]{%
  \underset{\substack{\uparrow\\\mathrlap{\text{\hspace{#3}#2}}}}{#1}}

\begin{document}
\maketitle

% problem 1
\begin{prob} 
	Let $G$ be a planar graph with no cycles of length $3$, $4$ or $5$. Show that $\chi(G) \le 3$.
\end{prob}

\begin{sol}
	We will prove the result by induction on $v(G)$. If $v(G) \le 3$, we are good!

	Let $G^*$ be a graph obtained by taking $G$ and adding edges between distinct connected components until it becomes connected. Since $G^*$ is planar and connected, Euler's formula for planar graphs implies $
		v(G^*) - e(G^*) + f(G^*) = 2.
	$

	Adding edges between distinct connected components does not create new faces, so $f(G) = f(G^*)$. Also, $e(G) \le e(G^*)$  and $v(G) = v(G^*)$; thus  \[
		v(G) - e(G) + f(G) \ge 2.
	\]

	By double conting edges, and using that each face has at least $6$ edges, we conclude \[
		2e(G) \ge 6f(G).
	\]

	The last two equations imply \[
		\sum_{v \in V(G)}\deg(v) = 2e(G) \ge 3(v(G) - 2),
	\]
	therefore there exists $u \in V(G)$ such that $\deg(u) \le 2$.

	Induction hypothesis implies that there exists a proper $3$-colouring of the vertices of $G - \{u\}$. Since $u$ has at most two neighbors, there exists a colour that does not appear on its neighbors. Define this as the colour of $u$. Now, we have a proper $3$-colouring of $G$; thus $\chi(G) \le 3$.
\end{sol}
\begin{thm}[Euler's formula for planar graphs]
	If $G$ is a planar and connected graph, $
		v(G) - e(G) + f(G) = 2$.
\end{thm}
\begin{dem}
	Induction on $e(G)$. If $e(G) = 0$, then $v(G) = 1$ and  $f(G) = 1$, so we are good! If $e(G) > 0$:

	\begin{enumerate}
		\item \textbf{If there is a leaf $v$,} then we define $G'$ by taking $v$ away.
			\begin{gather*}
				v(G') = v(G) - 1, \qquad
				e(G') = e(G) - 1, \qquad
				f(G') = f(G).     \\
				v(G') - e(G') + f(G') = 2 \implies v(G) - e(G) + f(G) = 2.
			\end{gather*}

		\item \textbf{If there is no leaf,} there is a cycle, then we define $G'$ by taking away an edge from the cycle.
			\begin{gather*}
				v(G') = v(G),     \qquad
				e(G') = e(G) - 1, \qquad
				f(G') = f(G) - 1. \\
				v(G') - e(G') + f(G') = 2 \implies v(G) - e(G) + f(G) = 2.
			\end{gather*}
	\end{enumerate}
\end{dem}


% problem 2
\newpage
\begin{prob}
	Prove that \[
		e(G) \ge \binom{\chi(G)}{2}.
	\]
\end{prob}
\begin{sol}
	We will prove, by induction on $v(G)$, that if $e(G) < \binom{r}{2}$, then there exists a proper $(r-1)$-colouring of the vertices of $G$.

	If $v(G) \le r-1$, we can assign each vertex a colour; so we are good!

	Suppose $v(G) \ge r$. Since \[\sum_{v \in V(G)}\deg(v) = 2e(G) < r(r-1),\] there exists a vertex $u$ such that $\deg(u) < r-1$.
	
	Induction hypothesis implies that there exists a proper $(r-1)$-colouring of the vertices of $G - \{u\}$. Since $u$ has at most $r-2$ neighbors, there exists a colour that does not appear on its neighbors. Define this as the colour of $u$. Now, we have a proper $(r-1)$-colouring of $G$.

	Wrapping everything up, by definition, there is no proper $(\chi(G) - 1)$-colouring of the vertices $G$, so \[
		e(G) \ge \binom{\chi(G)}{2}.
	\]
\end{sol}

% problem 3
\newpage
\begin{prob}
	Let $H$ be a $k$-uniform hypergraph with $m$ edges. Show that if $m < \frac{4^{k-1}}{3^k}$, then there exists a $4$-colouring of the vertices of $H$ such that every edge contains all four colours.
\end{prob}
\begin{sol}
	Pick a random $4$-colouring $c \colon V(H) \to \{R, G, B, Y\}$, with $\Prob(c(v) = R) = \Prob(c(v) = G) = \Prob(c(v) = B) = \Prob(c(v) = Y) = \frac{1}{4}$.

	For any $e \in E(H)$, \[
		\Prob(e \text{ does not have four colours}) \signexpl{\le}{union bound}{-2.5em} \binom{4}{3} \frac{3^k}{4^k} = \frac{3^k}{4^{k-1}}.
	\]

	Thus, if $m < \frac{4^{k-1}}{3^k}$, \[
		\Prob(\exists e \in E(H) : e \text{ does not have four colours}) \signexpl{\le}{union bound}{-2.5em} m \frac{3^k}{4^{k-1}} < 1.
	\]

	Finally, this implies that, with positive probability, all edges $e \in E(H)$ have four colours. Thus, there exists a $4$-colouring of the vertices of $H$ such that every edge contain all four colours.
\end{sol}

% problem 4
\newpage
\begin{prob}
	Use Van der Waerden's theorem to show that, for every $r$ and $k$, there exists a constant $\delta = \delta(r, k) > 0$ such that the following holds for all sufficiently large $n$:

	In every colouring of $\{1, \dots, n\}$ with $r$ colours, there are at least $\delta n^2$ monochromatic $k$-term arithmetic progressions.
\end{prob}
\begin{sol}
	Define $AP_x(a, d) := \{a, a+d, \dots, a+(x-1)d\}$.

	Let $T$ be the number of pairs $(S_1, S_2)$ such that:
	\begin{itemize}
		\item $S_1$ is monochromatic;
		\item $S_1 \subset S_2 \subset [n]$;
		\item $S_1 = AP_k(a, d)$, for some $a, d$;
		\item $S_2 = AP_{W(r, k)}(a', d')$, for some $a', d'$.
	\end{itemize}	

	Since each $S_1 = AP_{W(r, k)}(a', d')$ has size ${W(r, k)}$, it has a monochromatic $k$-term arithmetic progression. Thus \[
		T \ge \#\big(AP_{W(r, k)}(a', d')\big) = \#\big( (a', d') : a + ({W(r, k)}-1)d \le n\big) \ge \frac{n^2}{2{W(r, k)}}.
	\]

	On the other hand, for each $S_1 = AP_k(a, d)$, let's count the number of $S_2 = AP_{W(r, k)}(a', d')$ such that $S_1 \subset S_2$.
	Since $S_1 \subset S_2$, $d'$ divides $d$. Define $x := \frac{d}{d'} \le \frac{{W(r, k)}}{k}$. For a fixed positive integer $x \le \frac{{W(r, k)}}{k}$, there are at most ${W(r, k)}$ arithmetic progressions $S_2$ that work. 

	Thus, for a fixed $S_1$, there are at most $\frac{{W(r, k)}^2}{k}$ arithmetic progressions $S_2$ such that $S_1 \subset S_2$.

	Finally,  \[
		\frac{n^2}{2{W(r, k)}} \le T \le \frac{{W(r, k)}^2}{k} \cdot \#\big(\text{monochromatic }AP_k(a, d)\big),
	\]
	which implies that, for $\delta := \frac{k}{2{W(r, k)}^3}$,
	\[
		\#\big(\text{monochromatic }AP_k(a, d)\big) \ge \delta n^2.
	\]
\end{sol}

% problem 5
\newpage
\begin{prob}
	Prove that $r(K_r, T) = (r-1)(k-1) + 1$ for every tree $T$ with $k$ vertices.
\end{prob}
\begin{sol}
	We will prove the statement using induction on $r$.

	Consider the graph $G$ with $r-1$ disjoint red copies of $K_{k-1}$, and blue edges between any two vertices in distinct $K_{k-1}$. The graph $G$ contains no blue $K_r$ and no red $T$. Thus,  \[
		r(K_r, T) > (r-1)(k-1).
	\]

	Let $n \ge (r-1)(k-1) + 1$. We will show that any colouring of $K_n$ contains either a blue $K_r$ or a red $T$.

	Suppose there is a colouring of $K_n$ that does not have a blue $K_r$ or a red $T$.
	Let $N_B(v)$ and $N_R(v)$ denote the blue neighborhood and red neighborhood of $v$, respectively. There is no blue $K_{r-1}$ nor a red $T$ in $N_B(v)$. Thus, using the induction hypothesis, $|N_B(v)| \le (r-2)(k-1)$, which implies $|N_R(v)| \ge k-1$, for all $v$. By \cref{lem:treemindeg}, there exists a red copy of $T$ in $K_n$; a contradiction.

	Therefore,  \[
		r(K_r, T) \le (r-1)(k-1) + 1,
	\]
	which implies \[
		r(K_r, T) = (r-1)(k-1) + 1.
	\]
\end{sol}
\begin{lem}[Problem 3 from List 1]\label{lem:treemindeg}
	If $T$ is a tree with $k$ vertices and $G$ is a graph with minimum degree $k-1$, then $T \subset G$.
\end{lem}
\begin{dem}	
		We'll use induction on $k$. If $k = 1$, we're done!

		Pick a leaf $v$ of $T$. Its unique edge connects it to $u$. Let $T'$ be the tree without $v$. By induction, there is a copy $C_{T'}$ of $T'$ in $G$. Let $c_u$ be the copy of  $u$ in $C_{T'}$. Since $\deg(c_u) \leqslant k - 2$ in $C_{T'}$, but $\deg(c_u) \geqslant k-1$ in $G$, there is some vertex that is connected to $u$ outside of $C_{T'}$, say $c_v$. Thus, let $C_{T}$ be $C_{T'}$, adding $c_v$. $C_{T}$ is a copy of $T$ inside $G$.
\end{dem}

% problem 6
\newpage
\begin{prob}
	Show that, for every $p = p(n)$, \[
		\chi(G(n, p)) \ge \frac{pn}{2\log n},
	\]
	with high probability as $n \to \infty$.
\end{prob}
\begin{sol}
	Let $S \in \binom{[n]}{k}$, for $k \approx \frac{2\log n}{p}$. Then, \[
		\Prob(S\text{ is independent}) = (1-p)^{\binom{k}{2}}.
	\]

	Therefore,
	\begin{align*}
		\Prob\left(\exists S \in \binom{[n]}{k} : S\text{ is independent}\right)
		&\signexpl{\le}{union bound}{-2.5em} \binom{n}{k}(1-p)^{\binom{k}{2}}\\
		&\le \left(\frac{en}{k} \exp\left(-\frac{p(k-1)}{2}\right)\right)^k\\
		&\le \left(\frac{enp}{\log n} \exp\left(-\log n + \frac{p}{2}\right)\right)^k\\
		&\le \left(\frac{e^{3/2}}{\log n}\right)^k \to 0,\text{ as }n\to\infty.
	\end{align*}

	Thus, with high probability, there is no independent set of size $k \approx \frac{2\log n}{p}$, i.e., with high probability, \[
		\alpha(G(n, p)) \le \frac{2\log n}{p}.
	\]

	Since, for any graph $G$, $\chi(G) \alpha(G) \ge n$, we conclude that, with high probability,  \[
		\chi(G(n, p)) \ge \frac{np}{2\log n}.
	\]

\end{sol}

\begin{lem}
	For any graph $G$ with $n$ vertices, \[
		\chi(G) \alpha(G) \ge n.
	\]
\end{lem}
\begin{dem}
	By definition, there exists a partition of $V(G)$ into $\chi(G)$ independent sets. Each of those indepentent sets has size at most $\alpha(G)$. Therefore, \[
		\chi(G) \alpha(G) \ge n.
	\] 
\end{dem}

% problem 7
\newpage
\begin{prob}
	Let $r_k(H)$ be the smallest $n$ such that every colouring of $E(K_n)$ with $k$ colours contains a monochromatic copy of $H$. Show that \[
		k^{1+c} \le r_k(C_4) \le C\cdot k^2
	\]
	for some constants $C > c > 0$ and all sufficiently large $k$.
\end{prob}
\begin{sol}[for the upper bound]
	Suppose $c \colon E(K_n) \to [k]$ produces no monochromatic $C_4$. By counting monochromatic cherries, we conclude \[
		\frac{n(n-1)(n-k-1)}{2k} = \sum_v\left( k \binom{\frac{n-1}{k}}{2} \right) \le \sum_v \sum_{\text{colour }i} \binom{d_i(v)}{2} \signexpl{\le}{counting per vertex}{-4.5em} \#\text{monochromatic cherries} \signexpl{\le}{counting per pair}{-4em} \binom{n}{2} k.
	\]
	
	This implies $n \le k^2 + k + 1$. Therefore, for any fixed $C > 1$, for all sufficiently large $k$, \[
		r_k(C_4) \lesssim Ck^2.
	\]
\end{sol}
\begin{idea}[for the lower bound]
	Let $n = k^{1+\epsilon}$.

	Let's colour the edges of $K_n$ randomly and independently, with the probability of an edge being of a given colour is $\frac{1}{k}$. For a fixed colour, the graph of this colour is $G(n, \frac{1}{k})$.

	The expected number of $C_4$ of a given colour is
	\begin{align*}
		\Exp\left[\#\left(C_4 \text{ in } G\left(k^{1+\epsilon}, \frac{1}{k}\right)\right)\right]
		&\le \frac{1}{2} k^{4\epsilon}.
	\end{align*}

	Thus, the expected number of monochromatic $C_4$ is $\frac{1}{2}k^{1+4\epsilon}$.

	\emph{We cannot remove a vertex from each monochromatic $C_4$, because $\frac{1}{2}k^{1+4\epsilon} \gg k^{1+\epsilon}$.}
\end{idea}

% problem 8
\newpage
\begin{prob}
	Define \[
		\hat r(H) := \min \{e(G) : G \to H\},
	\]
	where $G \to H$ means that every two-colouring of the edges of $G$ contains a monochromatic copy of $H$.
	Prove that \[\hat r(K_t) = \binom{R(t)}{2},\] for every  $t \in \NN$.
\end{prob}
\begin{sk}
	Note that, by the definition of $R(t)$, $K_{R(t)} \to K_t$. Therefore, \[
		\hat r(K_t) \le \binom{R(t)}{2},
	\]
	for every $t \in \NN$.

	Conversely, let $G$ be a graph with $e(G) < \binom{R(t)}{2}$. We want to show that there exists a $2$-colouring of $E(G)$ that avoids a monochromatic copy of $K_t$.

	\begin{idea}[1]
		Let's try induction on $v(G)$.

		If $v(G) < R(t)$, then $G \subset K_{R(t) - 1}$. By definition, there exists a $2$-colouring of $E(K_{R(t)-1})$ that avoids monochromatic $K_t$; the restriction to $E(G)$ of this $2$-colouring also avoids monochromatic $K_t$.

		If $v(G) \ge R(t)$, then $\sum_{v} \deg(v) = 2e(G) < R(t)(R(t)-1)$. Thus, there exists $u$ such that $\deg(u) \le R(t) - 1$. Apply the induction hypothesis on $G - \{u\}$. \emph{Is there a smart way to colour the edges from $u$?}
	\end{idea}

	\begin{idea}[2]
		Let's pick a random $2$-colouring of $E(G)$. \emph{Maybe we can show $$\Prob(\text{monochromatic copy of } H \text{ in } G) < 1?$$}
	\end{idea}
\end{sk}

\end{document}
