\documentclass[10pt, a4paper]{article}
\usepackage[utf8]{inputenc}
\usepackage[english]{babel}
\usepackage{fullpage}
\usepackage{parskip}
\usepackage[prob-boxed]{zeus}

\DeclareMathOperator\ex{ex}

\title{\textbf{\textsf{Combinatorics I, List 1}}}
\author{\textsc{Guilherme Zeus Dantas e Moura}\\[2pt]\href{mailto:zeusdanmou@gmail.com}{\texttt{zeusdanmou@gmail.com}}}
\date{IMPA, Summer 2021}

\begin{document}
\maketitle

% PROBLEM 1
\begin{prob}
	By defining, and calculating the expectation of, a suitable random variable, show that every graph $G$ has a bipartite subgraph with at least $e(G)/2$ edges.
\end{prob}
\begin{sk}
	Let's pick a random bipartition $(A, B)$ of $V(G)$, with probability $1/2$ of $v \in A$, chosen independently.

	Define \[
		X = \#\left( a \sim b : a \in A \text{ and } b \in B\right).
	\]
\end{sk}

% PROBLEM 2
\newpage
\begin{prob}
	Show that $R(3, 4) \leqslant 9$, $R(4, 4) \leqslant 18$ and $R(3, 3, 3) \leqslant 17$.
\end{prob}
\begin{sol}
	
\end{sol}

% PROBLEM 3
\newpage
\begin{prob}
	Show that every graph of average degree $d$ contains a subgraph of minimum degree at least $d/2$. Deduce that  $\ex(n, T) \leqslant (k-1)n$ for every tree $T$ with $k$ vertices.
\end{prob}
\begin{sol}
	
\end{sol}

% PROBLEM 4
\newpage
\begin{prob}
	Show that if $T$ is a tree with $k$ vertices and $G$ is a graph with minimum degree $k-1$, then $T \subset G$. Deduce that $r(K_3, T) = 2k-1$.
\end{prob}
\begin{sk}
	Lower bound: Two blobs with $k-1$. Blue edge iff two vertices are in the same blob.

	Upper bound: Show that $d_R(v) \le k - 1$. Therefore,  $d_B(v) \ge k -1$.
\end{sk}

% PROBLEM 5
\newpage
\begin{prob}
	Let $T_1, \dots, T_k$ be subtrees of a tree $T$, any two of which have at least one vertex in common. Prove that there is a vertex in all the $T_i$.
\end{prob}
\begin{sk}
	Induction on $k$. Merge two trees $T_1, T_2 \subset T$ into one tree $T' \subset T$ that contains all vertices that were in $T_1$ and $T_2$.
\end{sk}

% PROBLEM 6
\newpage
\begin{prob}
	Let $R_r(3)$ denote the $r$-colour Ramsey number of a triangle. Show that \[
		2^r \leqslant R_r(3) \leqslant 3 \cdot r!.
	\]
	Show moreover that $R_r(3) \leqslant 5^{r/2}$.
\end{prob}
\begin{sk}
	Recursion. Pick a vertex and look to the blobs for each edge color.
\end{sk}

% PROBLEM 7
\newpage
\begin{prob}
	Let $g(n)$ be the largest integer such that there exists a graph with the following properties: $|V(G)| = n$, $e(G) = m$, and it is possible to red-blue colour the edges of $G$ without creating a monochromatic triangle.

	Show that $g(n)/\binom{n}{2}$ converges, and find $c$ such that $g(n)/\binom{n}{2} \to c$ as $n \to \infty$.
\end{prob}
\begin{sk}
Lower bound:
Turan's graph with 5 blobs.

Upper bound: If there are more edges than Turan with 5 blobs, then there is a $K_6$, which implies there is a monochromatic triangle.
\end{sk}

% PROBLEM 8
\newpage
\begin{prob}
	Recall that $\alpha(G)$ denotes the size of the largest independent set in $G$. Show that, for every graph $G$, \[
		\alpha(G) \geqslant \sum_{v \in v(G)} \frac{1}{d(v) + 1}.
	\]
\end{prob}
\begin{sk}
	Let $V(G) = [n].$
	Pick a random permutation $\pi \colon [n] \to [n]$. For each $v \in [n]$, define \[A_v := \{ u \in V(G) : u \sim v \text{ and } \pi(v) < \pi(u)\}.\]
\end{sk}

% PROBLEM 9
\newpage
\begin{prob}
	Let $C(s)$ be the smallest $n$ such that every connected graph on $n$ vertives has, as an \emph{induced} subgraph, either a complete $K_s$, a star $K_{1, s}$ or a path $P_s$ of lenfth $s$.

	Show that $C(s) \leqslant R(s)^s$, where $R(S)$ is the Ramsey number of $s$.
\end{prob}
\begin{sol}
	
\end{sol}

% PROBLEM 10
\newpage
\begin{prob}
	Prove that $R(3, k) \geqslant k^{1+c}$ for some $c > 0$.
\end{prob}
\begin{sk}
	\[
		X_k = \#(K_k \subset G(n, p))
	\]
	\[
		X_3 = \#(K_k \subset \overline{G(n, p)})
	\]
\end{sk}

% PROBLEM 11
\newpage
\begin{prob}
	Show that there is an infinite set $S$ of positive integers such that the sum of any two distinct elements of $S$ has an even number of distinct prime factors.
\end{prob}
\begin{sk}
	Let $S = \{n : n \equiv 1 \pmod{3}\}$.

	Let $x \sim y$ be blue iff $x + y$ has a even number of distinct prime factors. Ramsey's Theorem implies that there is an infinite. If it's blue, we're done! If it's red, then multiply every number by $3$. (a new prime factor!)
\end{sk}

\newpage
\begin{prob}
	Suppose we are given $n$ points and $n$ lines in the plane. Show that there are at most $O(n^{3/2})$ point-line incidences.
\end{prob}
\begin{sol}
	
\end{sol}

\end{document}
